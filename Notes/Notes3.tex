%%%%%%%%%%%%%%%%%%%%%%%%%%%%%%%%%%%%%%%%%%%%%%%%%%%%%%%%%%%%%%%%%%%%%%%%%%%%%%%%%%%%%%%%%%%%%%%%%%%%%%%%%%%%%%%%%%%%%%%%%%%%%%%%%%%%%%%%%%%%%%%%%%%%%%%%%%%%%%%%%%%
% Written By Michael Brodskiy
% Class: Cornerstone Engineering 1 & 2 (GE1501 & GE1502)
% Professor: B. O'Connell
%%%%%%%%%%%%%%%%%%%%%%%%%%%%%%%%%%%%%%%%%%%%%%%%%%%%%%%%%%%%%%%%%%%%%%%%%%%%%%%%%%%%%%%%%%%%%%%%%%%%%%%%%%%%%%%%%%%%%%%%%%%%%%%%%%%%%%%%%%%%%%%%%%%%%%%%%%%%%%%%%%%

\include{Includes.tex}

\title{Documentation}
\date{\today}
\author{Michael Brodskiy\\ \small Professor: B. O'Connell}

\begin{document}

\maketitle

\begin{itemize}

  \item Design Notebook

    \begin{itemize}

      \item Why Keep a Design Notebook?

        \begin{itemize}

          \item Documents/provides evidence of your work (both individually and as a group)

          \item Useful resource for preparing reports

          \item Serves as an object of memory

          \item Becomes a communication tool of past activity for new colleagues

        \end{itemize}

      \item Format

        \begin{itemize}

          \item Typically you'd use a physical tamper-proof notebook

          \item Front Matter

            \begin{itemize}

              \item Put your name(s) and a contact e-mail on cover

              \item Include a team contact page

              \item Leave space in the front for a table of contents

            \end{itemize}

          \item Inside

            \begin{itemize}

              \item Number pages

              \item Date all entries

            \end{itemize}

          \item Record Directly into the notebook

            \begin{itemize}

              \item As often as possible

              \item Take pics of notes, dry erase boards, etc.

              \item Add links to other documents/resources used

            \end{itemize}

          \item Include narrative to describe any sketches, diagrams, plots, and equations

            \begin{itemize}

              \item Don't assume you're the one reading this

              \item Someone else won't have the first-person experience that you do
                
            \end{itemize}

          \item Any and all work on your project

            \begin{itemize}

              \item Ideation, sketches, calculations, etc.

            \end{itemize}

          \item Project meeting notes

            \begin{itemize}

              \item Agendas, discussion items, and action items

            \end{itemize}

          \item Relevant external info

            \begin{itemize}

              \item Salient lectures and other reference material

            \end{itemize}

          \item Consultations

            \begin{itemize}

              \item Include contact information

            \end{itemize}

          \item Detailed notes on any compiled info

            \begin{itemize}

              \item Library and patent searched

              \item Observations

            \end{itemize}

          \item For digital content, describe the work done and where it can be found

          \item Include conclusions and recommendations 

          \item Narrative — Describe what is being done

            \begin{itemize}

              \item Enough that other engineers can follow

            \end{itemize}

          \item Not rigidly formatted

            \begin{itemize}

              \item More guidelines than rules

            \end{itemize}

          \item Not for everyone's consumption

            \begin{itemize}

              \item But should be readable by other engineers

              \item Someone at your level should be able to follow your EDP

            \end{itemize}

          \item Provides a time-lapse view of your engineering design process

        \end{itemize}

    \end{itemize}

  \item Status Memos

    \begin{itemize}

      \item This is a formal statement

        \begin{itemize}

          \item In this case, an update on your project status

        \end{itemize}

      \item They are a primary form of communication in many industries

        \begin{itemize}

          \item For informing colleagues of progress or status

          \item For making formal requests as part of your design activity

        \end{itemize}

      \item A memo is a brief communication written to busy people who want information as soon as possible

        \begin{itemize}

          \item Economize your ideas

          \item Be concise

        \end{itemize}

      \item What to include

        \begin{itemize}

          \item Introduction with the main point

            \begin{itemize}

              \item The header

                \begin{itemize}

                  \item To, From, Date, Subject, CC, Attachments

                \end{itemize}

              \item A brief summary

                \begin{itemize}

                  \item Necessary background

                  \item A clearly stated purpose

                  \item Preview of what's to be discussed

                \end{itemize}

            \end{itemize}

          \item Discussion of the topics

            \begin{itemize}

              \item A brief discussion of each of the main topics

              \item Approximately 1-2 paragraphs each

            \end{itemize}

          \item Review what's been said

            \begin{itemize}

              \item Restate the main points and associated main ideas

              \item Include follow-up contact information

            \end{itemize}

          \item Specific topics will be included as requirements in the project assignments

        \end{itemize}

      \item Know who your readers are

        \begin{itemize}

          \item Primary — ``To:''

          \item Secondary — ``CC:''

        \end{itemize}

      \item Know why you are writing this

        \begin{itemize}

          \item This should be clear to the reader

          \item Most difficult sentence is a statement of purpose

        \end{itemize}

    \end{itemize}

\end{itemize}

\end{document}

