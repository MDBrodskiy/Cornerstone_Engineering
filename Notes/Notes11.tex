%%%%%%%%%%%%%%%%%%%%%%%%%%%%%%%%%%%%%%%%%%%%%%%%%%%%%%%%%%%%%%%%%%%%%%%%%%%%%%%%%%%%%%%%%%%%%%%%%%%%%%%%%%%%%%%%%%%%%%%%%%%%%%%%%%%%%%%%%%%%%%%%%%%%%%%%%%%%%%%%%%%
% Written By Michael Brodskiy
% Class: Cornerstone Engineering 1 & 2 (GE1501 & GE1502)
% Professor: B. O'Connell
%%%%%%%%%%%%%%%%%%%%%%%%%%%%%%%%%%%%%%%%%%%%%%%%%%%%%%%%%%%%%%%%%%%%%%%%%%%%%%%%%%%%%%%%%%%%%%%%%%%%%%%%%%%%%%%%%%%%%%%%%%%%%%%%%%%%%%%%%%%%%%%%%%%%%%%%%%%%%%%%%%%

%%%%%%%%%%%%%%%%%%%%%%%%%%%%%%%%%%%%%%%%%%%%%%%%%%%%%%%%%%%%%%%%%%%%%%%%%%%%%%%%%%%%%%%%%%%%%%%%%%%%%%%%%%%%%%%%%%%%%%%%%%%%%%%%%%%%%%%%%%%%%%%%%%%%%%%%%%%%%%%%%%%
% Written By Michael Brodskiy
% Class: Cornerstone Engineering 1 & 2 (GE1501 & GE1502)
% Professor: B. O'Connell
%%%%%%%%%%%%%%%%%%%%%%%%%%%%%%%%%%%%%%%%%%%%%%%%%%%%%%%%%%%%%%%%%%%%%%%%%%%%%%%%%%%%%%%%%%%%%%%%%%%%%%%%%%%%%%%%%%%%%%%%%%%%%%%%%%%%%%%%%%%%%%%%%%%%%%%%%%%%%%%%%%%

\documentclass[12pt]{article} 
\usepackage{alphalph}
\usepackage[utf8]{inputenc}
\usepackage[russian,english]{babel}
\usepackage{titling}
\usepackage{amsmath}
\usepackage{graphicx}
\usepackage{enumitem}
\usepackage{amssymb}
\usepackage[super]{nth}
\usepackage{everysel}
\usepackage{ragged2e}
\usepackage{geometry}
\usepackage{multicol}
\usepackage{fancyhdr}
\usepackage{cancel}
\usepackage{siunitx}
\geometry{top=1.0in,bottom=1.0in,left=1.0in,right=1.0in}
\newcommand{\subtitle}[1]{%
  \posttitle{%
    \par\end{center}
    \begin{center}\large#1\end{center}
    \vskip0.5em}%

}
\usepackage{hyperref}
\hypersetup{
colorlinks=true,
linkcolor=blue,
filecolor=magenta,      
urlcolor=blue,
citecolor=blue,
}


\title{Evaluating a Design}
\date{\today}
\author{Michael Brodskiy\\ \small Professor: B. O'Connell}

\begin{document}

\maketitle

\begin{itemize}

  \item An Ongoing Process

    \begin{itemize}

      \item Goals of each phase

      \item Future directions in light of results

      \item Logic within decision points for fallacies

      \item Assumptions being made

      \item Data drives everything

    \end{itemize}

  \item Evaluation Checklist

    \begin{itemize}

      \item Is the solution logical?

      \item Are all the criteria and constraints satisfied?

      \item Does the proposed solution solve the real problem?

      \item Is this a permanent or temporary solution?

      \item Have you challenged the assumptions and information provided?

      \item Have you considered the potential problems?

      \item Have you argued both sides? Both the positive and negative impacts?

    \end{itemize}

  \item Engineering Research Design — A set of methods and procedures used in collecting and analyzing data deemed pertinent to understanding and acceptance of a solution's fitness for the resolution of a design problem

  \item Distinct Approaches

    \begin{itemize}

      \item Applied

        \begin{itemize}

          \item Moves towards finding a solution for an immediate problem

        \end{itemize}

      \item Fundamental

        \begin{itemize}

          \item Concentrates on the pure formulation of a theory

        \end{itemize}

      \item Confirmatory

        \begin{itemize}

          \item Seeks to test an apriori hypothesis

        \end{itemize}

      \item Exploratory

        \begin{itemize}

          \item Seeks to form a posteriori hypothesis

        \end{itemize}

    \end{itemize}

  \item Two Types of Research Design

    \begin{itemize}

      \item Quantitative — The objective empirical investigation of observable phenomena via outcome oriented statistical, mathematical, or computational techniques

        \begin{itemize}

          \item Level of occurrence

          \item Asks ``how many?''

          \item Studies the event

          \item Objective

          \item Discovery and proof

          \item More definitive

          \item Describes

          \item Scalable

        \end{itemize}

      \item Qualitative — The subjective and holistic investigation to explain and gain understanding and insight pertaining to phenomena via ongoing observation and interaction with the subject

        \begin{itemize}

          \item Depth of understanding

          \item Asks ``why that many?''

          \item Studies motivation

          \item Subjective

          \item Enables discovery

          \item Exploratory in nature

          \item Interprets

          \item Varies case to case

        \end{itemize}

    \end{itemize}

  \item Interview Protocols

    \begin{itemize}

      \item Asking the right questions to get at the heart of an issue is a skill that sets critical thinkers apart from others 

      \item Two Basic Categories:

        \begin{itemize}

          \item Convergent

            \begin{itemize}

              \item These questions aim for a finite range of acceptably accurate responses

              \item Seeking comprehension, application of information, or further defined analysis

            \end{itemize}

          \item Divergent

            \begin{itemize}

              \item These explore different avenues, variations, alternate possibilities and scenarios

              \item Correctness based more on logical projection, context, and conjecture

            \end{itemize}

        \end{itemize}

    \end{itemize}

  \item Observation Protocols

    \begin{itemize}

      \item Observation Aids

        \begin{itemize}

          \item Tally Clickers

          \item Video Data

            \begin{itemize}

              \item Continuous

              \item Time Lapse

            \end{itemize}

          \item Audio Data

            \begin{itemize}

              \item General

              \item Multi-location

            \end{itemize}

          \item System Tracking

            \begin{itemize}

              \item Embedded tracking software

              \item Screen recording

            \end{itemize}

          \item AI Programs

            \begin{itemize}

              \item Analyzing the data

            \end{itemize}

        \end{itemize}

      \item Observable Phenomena

        \begin{itemize}

          \item Interactions

            \begin{itemize}

              \item Interpersonal

              \item Inter-technological

              \item Combinations

            \end{itemize}

          \item Emotional Responses

          \item Pre/Post

          \item Lack of Activity

        \end{itemize}

    \end{itemize}

\end{itemize}

\end{document}

