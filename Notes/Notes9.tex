%%%%%%%%%%%%%%%%%%%%%%%%%%%%%%%%%%%%%%%%%%%%%%%%%%%%%%%%%%%%%%%%%%%%%%%%%%%%%%%%%%%%%%%%%%%%%%%%%%%%%%%%%%%%%%%%%%%%%%%%%%%%%%%%%%%%%%%%%%%%%%%%%%%%%%%%%%%%%%%%%%%
% Written By Michael Brodskiy
% Class: Cornerstone Engineering 1 & 2 (GE1501 & GE1502)
% Professor: B. O'Connell
%%%%%%%%%%%%%%%%%%%%%%%%%%%%%%%%%%%%%%%%%%%%%%%%%%%%%%%%%%%%%%%%%%%%%%%%%%%%%%%%%%%%%%%%%%%%%%%%%%%%%%%%%%%%%%%%%%%%%%%%%%%%%%%%%%%%%%%%%%%%%%%%%%%%%%%%%%%%%%%%%%%

%%%%%%%%%%%%%%%%%%%%%%%%%%%%%%%%%%%%%%%%%%%%%%%%%%%%%%%%%%%%%%%%%%%%%%%%%%%%%%%%%%%%%%%%%%%%%%%%%%%%%%%%%%%%%%%%%%%%%%%%%%%%%%%%%%%%%%%%%%%%%%%%%%%%%%%%%%%%%%%%%%%
% Written By Michael Brodskiy
% Class: Cornerstone Engineering 1 & 2 (GE1501 & GE1502)
% Professor: B. O'Connell
%%%%%%%%%%%%%%%%%%%%%%%%%%%%%%%%%%%%%%%%%%%%%%%%%%%%%%%%%%%%%%%%%%%%%%%%%%%%%%%%%%%%%%%%%%%%%%%%%%%%%%%%%%%%%%%%%%%%%%%%%%%%%%%%%%%%%%%%%%%%%%%%%%%%%%%%%%%%%%%%%%%

\documentclass[12pt]{article} 
\usepackage{alphalph}
\usepackage[utf8]{inputenc}
\usepackage[russian,english]{babel}
\usepackage{titling}
\usepackage{amsmath}
\usepackage{graphicx}
\usepackage{enumitem}
\usepackage{amssymb}
\usepackage[super]{nth}
\usepackage{everysel}
\usepackage{ragged2e}
\usepackage{geometry}
\usepackage{multicol}
\usepackage{fancyhdr}
\usepackage{cancel}
\usepackage{siunitx}
\geometry{top=1.0in,bottom=1.0in,left=1.0in,right=1.0in}
\newcommand{\subtitle}[1]{%
  \posttitle{%
    \par\end{center}
    \begin{center}\large#1\end{center}
    \vskip0.5em}%

}
\usepackage{hyperref}
\hypersetup{
colorlinks=true,
linkcolor=blue,
filecolor=magenta,      
urlcolor=blue,
citecolor=blue,
}


\title{Implementation}
\date{\today}
\author{Michael Brodskiy\\ \small Professor: B. O'Connell}

\begin{document}

\maketitle

\begin{itemize}

  \item Proof of Concept

    \begin{itemize}

      \item Functionally equivalent but incomplete representations tested in a controlled environment

      \item Model certain relevant characteristics and apply those models to potential designs

    \end{itemize}

  \item Models

    \begin{itemize}

      \item Models are designed to behave in the same way as the real processes or systems but under certain conditions

    \end{itemize}

  \item Prototypes

    \begin{itemize}

      \item A prototype is the first of its kind

      \item Prototypes are used to demonstrate that a product will function as designed, tested in actual/uncontrolled/real world operating environments

    \end{itemize}

  \item What does your first prototype look like?

    \begin{itemize}

      \item A proof of concept 

      \item Depends on your design, budget, goals

      \item For the initial demo, use household items to show the functionality of your design

    \end{itemize}

  \item Refining your Prototype

    \begin{itemize}

      \item Mock-ups

        \begin{itemize}

          \item Construct a mock-up of a 3D part from 2D cutouts

          \item These 2D parts can be made using vinyl cutter or a laser cutter

          \item Parts are then assembled into 3D mock-ups of design

          \item Materials used for these mock-ups might be foam, thin plastic, or wood

        \end{itemize}

      \item Machining

        \begin{itemize}

          \item Woodworking machines

          \item Metal shop

        \end{itemize}

      \item Rapid Prototyping

        \begin{itemize}

          \item A group of techniques used to fabricate a scale model of a physical part or assembly using three-dimensional computer-aided design (CAD) data

          \item 3D printing is one of those techniques: Convert these 3D files into thin 2D layers to build the 3D part

        \end{itemize}

    \end{itemize}

  \item Planning

    \begin{itemize}

      \item Determine the Critical Path

        \begin{itemize}

          \item List your goals

          \item Identify the main tasks

          \item Identify subtasks

        \end{itemize}

      \item Evaluate tasks

        \begin{itemize}

          \item Estimate durations

          \item Estimate support

          \item Estimate materials

        \end{itemize}

      \item Organize those tasks

        \begin{itemize}

          \item Allocate the time needed

          \item Assign to team members

          \item Budget for materials

        \end{itemize}

    \end{itemize}

  \item Industry Barriers to Implementation

    \begin{itemize}

      \item Manufacturing costs

      \item Maintenance costs

      \item Lack of suitable materials

      \item Unreasonable labor demands

      \item Unsustainable practices

    \end{itemize}

  \item Design for X

    \begin{itemize}

      \item X can be:

        \begin{itemize}

          \item Manufacturing

          \item Sustainability

          \item Reliability

          \item Quality

          \item Maintainability

          \item Disassembly

          \item Recyclability

        \end{itemize}

    \end{itemize}

  \item Commonly Added Objectives

    \begin{itemize}

      \item Minimize cost to us

      \item Minimize price for consumer

      \item Decrease time to market

      \item Minimize maintenance and repair costs

      \item Maximize level of quality and reliability

      \item Make the design upgradable

    \end{itemize}

  \item Design for Manufacturing Guidelines

    \begin{itemize}

      \item Use standard parts

      \item Design for an assembly line

        \begin{itemize}

          \item Use machine assembly

          \item Minimize number of assembly operations

          \item Minimize tolerances for easy assembly

          \item Provide access points for assembly 

        \end{itemize}

      \item Minimize number of parts

        \begin{itemize}

          \item Design parts to be multifunctional

          \item Minimize part variations

          \item Use reusable models and subassemblies

          \item Avoid separate fasteners

        \end{itemize}

    \end{itemize}

  \item Design for Sustainability

    \begin{itemize}

      \item Overall Design

        \begin{itemize}

          \item Dematerialize the product

          \item Use materials with low resource requirements

        \end{itemize}

      \item Manufacturing

        \begin{itemize}

          \item Reduce process resource consumption

          \item Reduce process emissions

          \item Consider process and background and foreground supply chains

          \item Dematerialize the product chain

          \item Improve factory safety and ergonomics

          \item Improve factory economics

        \end{itemize}

    \end{itemize}

\end{itemize}

\end{document}

