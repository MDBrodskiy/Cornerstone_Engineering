%%%%%%%%%%%%%%%%%%%%%%%%%%%%%%%%%%%%%%%%%%%%%%%%%%%%%%%%%%%%%%%%%%%%%%%%%%%%%%%%%%%%%%%%%%%%%%%%%%%%%%%%%%%%%%%%%%%%%%%%%%%%%%%%%%%%%%%%%%%%%%%%%%%%%%%%%%%%%%%%%%%
% Written By Michael Brodskiy
% Class: Cornerstone Engineering 1 & 2 (GE1501 & GE1502)
% Professor: B. O'Connell
%%%%%%%%%%%%%%%%%%%%%%%%%%%%%%%%%%%%%%%%%%%%%%%%%%%%%%%%%%%%%%%%%%%%%%%%%%%%%%%%%%%%%%%%%%%%%%%%%%%%%%%%%%%%%%%%%%%%%%%%%%%%%%%%%%%%%%%%%%%%%%%%%%%%%%%%%%%%%%%%%%%

\include{Includes.tex}

\title{Solution Generation}
\date{\today}
\author{Michael Brodskiy\\ \small Professor: B. O'Connell}

\begin{document}

\maketitle

\begin{itemize}

  \item Barriers to Idea Generation

    \begin{itemize}

      \item Perception Blocks

        \begin{itemize}

          \item Stereotyping

          \item Limiting the Problem (unnecessarily)

          \item Information Overload

        \end{itemize}

      \item Emotional Blocks

        \begin{itemize}

          \item Fear of Risk Taking

            \begin{itemize}

              \item Stems from childhood

            \end{itemize}

          \item Lack of an Appetite for Chaos

            \begin{itemize}

              \item Learn to live with confusion

            \end{itemize}

          \item Judging while generating ideas

            \begin{itemize}

              \item Negativity

            \end{itemize}

          \item Lack of Challenge

            \begin{itemize}

              \item Too easy to take on

            \end{itemize}

          \item Thinking all of some of the problem cannot be solved

            \begin{itemize}

              \item Lack of energy

            \end{itemize}

          \item Inability to Incubate

            \begin{itemize}

              \item Rushing!

            \end{itemize}

        \end{itemize}

      \item Cultural Blocks

        \begin{itemize}

          \item Imposed by our immediate social or physical environment

        \end{itemize}

      \item Expressive Blocks

        \begin{itemize}

          \item Inability to Communicate

          \item Information goes unstated or unknown

          \item Can not build off of thoughts not presented

        \end{itemize}

      \item Environmental Blocks

        \begin{itemize}

          \item Distractions that inhibit deep, prolonged concentration

            \begin{itemize}

              \item Other priorities

              \item Phones

              \item People

            \end{itemize}

        \end{itemize}

      \item Intellectual Blocks

        \begin{itemize}

          \item Inflexible or inadequate uses of problem solving strategies

          \item Lack of intellectual skills necessary

          \item Lack of information

          \item Expanding your mind

            \begin{itemize}

              \item Conduct literature reviews

              \item Conduct a patent search

              \item Benchmark existing products

              \item Reverse engineer devices

              \item Consult an expert

            \end{itemize}

        \end{itemize}

    \end{itemize}

  \item Comments that Reduce Creativity and Ideation

    \begin{itemize}

      \item That won't work

      \item That's too radical

      \item It's not our job

      \item We don't have enough time

      \item That's too much hassle

      \item It's against our policy

      \item We haven't done it that way before

      \item That's too expensive

      \item That's not practical

      \item We can't solve this problem

    \end{itemize}

  \item Other Peoples' Views

    \begin{itemize}

      \item Think about walking around on your knees

        \begin{itemize}

          \item How would this change your perspective?

            \begin{itemize}

              \item Imagine the playground from a child's height

            \end{itemize}

        \end{itemize}

      \item What was your favorite playground toy?

        \begin{itemize}

          \item How could this be mimicked with used auto parts?

        \end{itemize}

    \end{itemize}

  \item Morphological Charts

    \begin{itemize}

      \item Configured as a matrix:

        \begin{itemize}

          \item Left most column is design goals (Objectives, constraints, and functions)

          \item Other columns are ways to achieve those features

        \end{itemize}

      \item Try to keep all features to the same level of detail

      \item Once all features and solutions are complete, choose combinations until design is created

    \end{itemize}

  \item The C-Sketch Method

    \begin{itemize}

      \item Team-based design strategy

      \item Can be more difficult with large teams

      \item Excellent for developing visual elements

      \item Everyone draws out some kind of design, this design is handed around and each person makes comments back on each person's design

    \end{itemize}

  \item The Gallery Method

    \begin{itemize}

      \item Team-based design strategy

      \item All team members create sketches within a time limit

      \item Sketches are all posted/shared together

      \item All sketches are discussed and critiqued

    \end{itemize}

  \item The Revision Method

    \begin{itemize}

      \item Improve an existing product or process without starting over

      \item Use Nth generation design to add or modify features

      \item Repurpose the design to meet new customer needs

    \end{itemize}

  \item Benchmarking and Best Practics

    \begin{itemize}

      \item Benchmarking — Compare your own product or process against a competitor's in order to improve

        \begin{itemize}

          \item How have other companies solved similar problems? How can I solve my problem?

          \item What do other companies do better than us?

        \end{itemize}

      \item Best Practices — Determine the best methods and techniques within your industry

        \begin{itemize}

          \item Research and gather the best information

          \item Incorporate the best methods into your operations

        \end{itemize}

    \end{itemize}

  \item Patent Search

    \begin{itemize}

      \item What is the state of the art?

      \item Find working designs

    \end{itemize}

  \item Bionics/Biomimetics — Search for solutions in nature

    \begin{itemize}

      \item Look in the natural world for inspiration

      \item Animals, plants, human body

      \item Examples: Velcro, sharkskin, geckos and lizards, solar cells

    \end{itemize}

  \item The Brainstorming Process

    \begin{itemize}

      \item Lateral Thinking

        \begin{itemize}

          \item Random stimulation

          \item Unrelated ideas

        \end{itemize}

      \item Vertical Thinking

        \begin{itemize}

          \item SCAMPER (\textbf{s}ubstitute, \textbf{c}ombine, \textbf{a}dapt, \textbf{m}odify, \textbf{p}ut to other uses, \textbf{e}liminate, \textbf{r}earrange) checklist

        \end{itemize}

    \end{itemize}

  \item Inversion Techniques

    \begin{itemize}

      \item Think of how to do the opposite

        \begin{itemize}

          \item Invert/reverse the problem statement

        \end{itemize}

    \end{itemize}

  \item Design Heuristic Cards

\end{itemize}

\end{document}

