%%%%%%%%%%%%%%%%%%%%%%%%%%%%%%%%%%%%%%%%%%%%%%%%%%%%%%%%%%%%%%%%%%%%%%%%%%%%%%%%%%%%%%%%%%%%%%%%%%%%%%%%%%%%%%%%%%%%%%%%%%%%%%%%%%%%%%%%%%%%%%%%%%%%%%%%%%%%%%%%%%%
% Written By Michael Brodskiy
% Class: Cornerstone Engineering 1 & 2 (GE1501 & GE1502)
% Professor: B. O'Connell
%%%%%%%%%%%%%%%%%%%%%%%%%%%%%%%%%%%%%%%%%%%%%%%%%%%%%%%%%%%%%%%%%%%%%%%%%%%%%%%%%%%%%%%%%%%%%%%%%%%%%%%%%%%%%%%%%%%%%%%%%%%%%%%%%%%%%%%%%%%%%%%%%%%%%%%%%%%%%%%%%%%

\include{Includes.tex}

\title{Programming}
\date{\today}
\author{Michael Brodskiy\\ \small Professor: B. O'Connell}

\begin{document}

\maketitle

\begin{itemize}

  \item Why is Programming Hard?

    \begin{itemize}

      \item Programming is Detail Oriented

        \begin{itemize}

          \item Is your syntax right?

            \begin{itemize}

              \item Did you forget a semi-colon?

              \item Is your capitalization correct?

            \end{itemize}

          \item Do your instructions make sense?

            \begin{itemize}

              \item Are they in the right order?

              \item Does that input exist yet?

            \end{itemize}

        \end{itemize}

      \item There is no room for mistakes

        \begin{itemize}

          \item You will make mistakes

          \item Learn from them though

        \end{itemize}

      \item A computer will do exactly what you tell it to

        \begin{itemize}

          \item Nothing more, nothing less
            
          \item EXACTLY what the code says

        \end{itemize}

      \item They do not understand ambiguity

        \begin{itemize}

          \item They will never infer anything

          \item They can't read minds

        \end{itemize}

      \item They do not make mistakes

    \end{itemize}

  \item Algorithmic Thinking

    \begin{itemize}

      \item Creating instructions for a computer

      \item A computer program is an algorithm written using the syntax of a programming language

      \item Algorithms are self-contained, step-by-step sets of operations to be performed

    \end{itemize}

  \item Algorithms

    \begin{itemize}

      \item A process or set of rules to be followed in calculations or other problem-solving operations, especially by a computer

    \end{itemize}

  \item Algorithmic Efficiency

    \begin{itemize}

      \item Not all algorithms are created equal

      \item The most efficient way may not be the most obvious

      \item In this class, though, do not consider efficiency when writing programs

        \begin{itemize}

          \item Stick with the most obvious algorithm

        \end{itemize}

    \end{itemize}

  \item Pseudocode

    \begin{itemize}

      \item An informal version of a program

      \item Acts like an outline for your formal code

        \begin{itemize}

          \item Used to help create formal code

          \item Typically becomes comments

        \end{itemize}

      \item Steps are in sequence, like in code

    \end{itemize}

  \item The 5 Steps to Problem Solving

    \begin{itemize}

      \item State problem clearly

      \item Identify inputs/outputs

      \item Write a flowchart or pseudocode

      \item Write code based on flowchart or pseudocode

      \item Test program with a variety of data

        \begin{itemize}

          \item Learn from mistakes and repeat if necessary

        \end{itemize}

    \end{itemize}

  \item Programming Concepts

    \begin{itemize}

      \item Data Handling

        \begin{itemize}

          \item Value

            \begin{itemize}

              \item A representation of a specific data type

            \end{itemize}

          \item Variable

            \begin{itemize}

              \item The storage location of that information

                \begin{itemize}

                  \item Typically limits and defines the data type allowed

                \end{itemize}

            \end{itemize}

          \item Data Types

            \begin{itemize}

              \item Integers — Whole numbers, limited by the space set aside (signed or unsigned)

              \item Floating Points — Called ``floats'', have decimal places

              \item Boolean — True or false

              \item Characters — Single characters, denoted by single quotes (every character has a corresponding number representation)

              \item Data Grouping

                \begin{enumerate}

                  \item Scalar — A single data unit

                  \item Array — A unidimensional row or column of data, also known as a vector

                  \item Matrix — A two-dimensional grouping of data

                \end{enumerate}

              \item Strings — Multiple characters, denoted by double quotes (an array of characters)

              \item Space Allocation — Size Matters

              \item Bit — Binary Digit (0 or 1)

              \item Byte — 8 Bits (all data is composed of bytes)

            \end{itemize}

        \end{itemize}

      \item Operators

        \begin{itemize}

          \item Arithmetic

            \begin{itemize}

              \item +, -, *, /

              \item $\wedge$, !

            \end{itemize}

          \item Trigonometry

          \item Logarithmic

          \item ``='' refers to assignment, ``=='' refers to equality

        \end{itemize}

      \item Control Structures

        \begin{itemize}

          \item Branching

            \begin{itemize}

              \item IFTTT — If, If/Else, If/Else If/Else; Action occurs dependent on a conditional statement or combination of conditionals

            \end{itemize}

          \item Looping

            \begin{itemize}

              \item While — Executes a block of code while a condition is true; checks for condition, if true, runs a block of code until condition is not true

              \item Do/While — Executes a block of code, then checks if a condition is true before repeating; runs a block of code, then checks for condition, repeats until condition is not true

              \item For — Executes a block of code repeatedly, using a variable with a predetermined set of values to cycle through; starts the loop with a variable (set to initial value), runs the block of code with that variable and value, runs the block of code with that variable and value, repeats until no more values are available, or a condition regarding the values is no longer true

            \end{itemize}

        \end{itemize}

      \item Data Structures

      \item Syntax

      \item Tools

    \end{itemize}

\end{itemize}

\end{document}

