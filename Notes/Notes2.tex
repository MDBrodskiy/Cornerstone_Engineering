%%%%%%%%%%%%%%%%%%%%%%%%%%%%%%%%%%%%%%%%%%%%%%%%%%%%%%%%%%%%%%%%%%%%%%%%%%%%%%%%%%%%%%%%%%%%%%%%%%%%%%%%%%%%%%%%%%%%%%%%%%%%%%%%%%%%%%%%%%%%%%%%%%%%%%%%%%%%%%%%%%%
% Written By Michael Brodskiy
% Class: Cornerstone Engineering 1 & 2 (GE1501 & GE1502)
% Professor: B. O'Connell
%%%%%%%%%%%%%%%%%%%%%%%%%%%%%%%%%%%%%%%%%%%%%%%%%%%%%%%%%%%%%%%%%%%%%%%%%%%%%%%%%%%%%%%%%%%%%%%%%%%%%%%%%%%%%%%%%%%%%%%%%%%%%%%%%%%%%%%%%%%%%%%%%%%%%%%%%%%%%%%%%%%

\include{Includes.tex}

\title{Defining Problems}
\date{\today}
\author{Michael Brodskiy\\ \small Professor: B. O'Connell}

\begin{document}

\maketitle

\begin{itemize}

  \item What is a problem statement?

    \begin{itemize}

      \item A problem statement is a clear, concise description of the issue(s) that need(s) to be addressed by a problem solving team

        \begin{itemize}

          \item Do \underline{not} write about multiple problems, tackle one thing at a time

          \item No broad or overly ambitious problems (break into components)

        \end{itemize}

      \item Should answer the following questions:

        \begin{itemize}

          \item What is the issue?

          \item Is there a need for this issue to be addressed?

          \item Why is this issue worth my attention?

        \end{itemize}

      \item Problem statements are important communication tools with the customer and among the team

    \end{itemize}

  \item What can a good problem statement do?

    \begin{itemize}

      \item Align efforts towards a common goal

      \item Define that goal

      \item Establish value in the goal

    \end{itemize}

  \item Bad Problem Statements

    \begin{itemize}

      \item Too precise

      \item Too vague

    \end{itemize}

  \item Clients rarely provide well-stated problems

  \item A good problem statement provides useful information to guide your design process:

    \begin{itemize}

      \item The issue

      \item The user

      \item The real need

      \item Why we care

      \item Form and function

      \item Objective

      \item Constraints

    \end{itemize}

  \item Errors, Biases, and Implied Solutions

    \begin{itemize}

      \item Errors — Incorrect information, faulty or incomplete data, or even simple mistakes

      \item Biases — Presumptions about the situation that may prove inaccurate because the client or the users may not fully grasp the entire situation

        \begin{itemize}

          \item \textit{Ex. ``Design a bicycle to transport four people on city streets''}

            \begin{itemize}

              \item Bicycle is limiting

              \item City streets is vague

              \item And more\ldots

            \end{itemize}

        \end{itemize}

      \item Implied Solution — The client's best and current guess at the answer; these frequently appear in a problem statement

        \begin{itemize}

          \item \textit{Ex. ``Develop a material that is able to withstand the extremely high temperatures of space capsule re-entry''}

              \begin{itemize}

                \item Rewritten: ``Protect the astronauts during re-entry into Earth's atmosphere''

              \end{itemize}

        \end{itemize}

    \end{itemize}

  \item Understanding Stakeholders

    \begin{itemize}

      \item User

        \begin{itemize}

          \item A person who will operate what is designed

        \end{itemize}

      \item Client

        \begin{itemize}

          \item A person or group or company that wants a design, usually to solve an existing problem

        \end{itemize}

      \item Engineer

        \begin{itemize}

          \item Hired by the client to find a solution to the problem

        \end{itemize}

    \end{itemize}

  \item Understanding Objectives, Functions, and Constraints

    \begin{itemize}

      \item Often requires asking a series of questions

      \item These lead to lists of desired attributes

      \item Objective — A feature or behavior that the design should have or exhibit

      \item Function — Those things that a designed device or system is supposed to do

      \item Constraint — A limit or restriction on the design's behaviors and attributes

    \end{itemize}

  \item Cause and Effect (Fishbone)

    \begin{itemize}

      \item Identify potential factors causing an issue

    \end{itemize}

  \item A very common tool

    \begin{itemize}

      \item Service (4 S's)

        \begin{itemize}

          \item Surrounding

          \item Supplies

          \item Systems

          \item Skills

        \end{itemize}

      \item Mfg (5 M's)

        \begin{itemize}

          \item Measurements

          \item Materials

          \item Manpower

          \item Methods

          \item Machines

        \end{itemize}

      \item Product (5\footnote{Was initially 5, number has still not changed despite an increase in list size} P's)

        \begin{itemize}

          \item Product (or service)

          \item Price

          \item Promotion

          \item Place

          \item Process
            
          \item People (personnel)

          \item Physical evidence

          \item Performance
            
        \end{itemize}

    \end{itemize}

  \item Fresh Eye Approach

    \begin{itemize}

      \item Explain the initial problem to someone outside of your design team

        \begin{itemize}

          \item Provides a new perspective

          \item Identifies what aspects draw attention

          \item What may be given too much attention

        \end{itemize}
        
    \end{itemize}

  \item Kepner-Tregoe (KT) Approach

    \begin{itemize}

      \item Seeks to reveal four dimensions of the problem:

        \begin{itemize}

          \item Identify — What?

          \item Timing — When?

          \item Location — Where?

          \item Magnitude — How Much?

        \end{itemize}

    \end{itemize}

  \item Duncker Diagram

    \begin{itemize}

      \item Present State $\rightarrow$ Desired State

        \begin{itemize}

          \item General Solutions which ``Make it OK not to\ldots''

          \item Functional Solutions

          \item Specific Solutions

        \end{itemize}

    \end{itemize}

\end{itemize}

\end{document}

