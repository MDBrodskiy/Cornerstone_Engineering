%%%%%%%%%%%%%%%%%%%%%%%%%%%%%%%%%%%%%%%%%%%%%%%%%%%%%%%%%%%%%%%%%%%%%%%%%%%%%%%%%%%%%%%%%%%%%%%%%%%%%%%%%%%%%%%%%%%%%%%%%%%%%%%%%%%%%%%%%%%%%%%%%%%%%%%%%%%%%%%%%%%
% Written By Michael Brodskiy
% Class: Cornerstone Engineering 1 & 2 (GE1501 & GE1502)
% Professor: B. O'Connell
%%%%%%%%%%%%%%%%%%%%%%%%%%%%%%%%%%%%%%%%%%%%%%%%%%%%%%%%%%%%%%%%%%%%%%%%%%%%%%%%%%%%%%%%%%%%%%%%%%%%%%%%%%%%%%%%%%%%%%%%%%%%%%%%%%%%%%%%%%%%%%%%%%%%%%%%%%%%%%%%%%%

\include{Includes.tex}

\title{Question 7, Design Exam}
\date{\today}
\author{Michael Brodskiy\\ \small Professor: B. O'Connell}

\begin{document}

\maketitle

\begin{center}
  \begin{tabular}[h!]{| c | c | c |}
    \hline
    Musts: & Low Energy Requirement & Produce a Compacted Can\\
    \hline
    Design & Go/No Go & Go/No Go\\
    Options & & \\
    \hline
    Option 1 & No Go & Go\\
    \hline
    Option 2 & Go & Go\\
    \hline
    Option 3 & Go & Go \\
    \hline
    Option 4 & Go & Go\\
    \hline
  \end{tabular}\\
  \vspace{10pt}

  \begin{tabular}[h!]{| c | c | c |}
    \hline
    & 100 & Safety \\
    Critical Objectives & 90 & \\
    & 80 &  \\
    & 70 & Durability \\
    \hline
    & 60 &  Ease of Use\\
    Significant Objectives & 50 & Few Moving Parts\\
    & 40 &   Cost\\
    & 30 & \\
    \hline
    & 20 &  Standard Parts\\
    Non-compulsory Objectives & 10 & \\
    & 0 &  \\
    \hline
  \end{tabular}\\
  \vspace{10pt}

  \begin{tabular}[h!]{| c | c | c | c | c | c | c | c |}
    \hline
    Wants: & Safety & Ease of Use & Durability & Parts & Cost & Few Moving Parts & \\
    \hline
    Designs: &  &  &  &  &  &  & Totals\\
    \hline
    \sout{Option 1:} &  &  &  &  &  &  & \\
    \hline
    Option 2: &  6 / 600 &  9 / 540  & 9 / 630 & 7 / 140 & 5 / 200 & 6 / 300  & 2400\\
    \hline
    Option 3: &  2 / 200 & 9 / 540  & 7 / 490  & 9 / 180  & 7 / 280 & 8 / 400 & 2090 \\
    \hline
    Option 4: &  9 / 900 & 9 / 540 & 9 / 630  & 9 / 180 & 5 / 200  & 8 / 400  & 2850\\
    \hline
  \end{tabular}
\end{center}

First and foremost, option 1 wasn't considered because it does not meet the ``low energy'' must. This is because, due to kids being weaker than full-grown adults, it is necessary for them to be able to operate this machine. Option 1 is the only option that does not somehow integrate gravity into the design, and, as such, it would be reliant on a very strong spring, which can also be a safety hazard. In terms of ranking, safety is above all critical, as we do not want kids crushing their digits. Standard parts aren't very important, and, as such, they end up being at the bottom. The other goals are a bit in the middle with some, like durability, being slightly more important. Option 4 is easily the most safe because, unlike option 3, which has a great potential to crush a finger or hand, 4 is operated by hand, limiting the possibility of crushing a hand. Option 2 is slightly safer, but it appears that there is a possibility of getting a foot stuck in the design, which can be dangerous if the spring brings the plate back. In general, aside from safety, all designs are fairly intuitive, but 3 is slightly less durable because of the string component. All use fairly easy-to-get parts, but 2 might be slightly harder to get parts for because of the ``two rail system''. Although 3 has two rails, it is much easier to get a string than a strong spring, so it redeems itself. The cost of all might be a bit, as obtaining a strong spring, a strong plate, or a strong lever might be difficult. All have fairly few moving parts, but option 4 seems to be a clear winner of the KTDA rating.

\end{document}

