%%%%%%%%%%%%%%%%%%%%%%%%%%%%%%%%%%%%%%%%%%%%%%%%%%%%%%%%%%%%%%%%%%%%%%%%%%%%%%%%%%%%%%%%%%%%%%%%%%%%%%%%%%%%%%%%%%%%%%%%%%%%%%%%%%%%%%%%%%%%%%%%%%%%%%%%%%%%%%%%%%%
% Written By Michael Brodskiy
% Class: Cornerstone Engineering 1 & 2 (GE1501 & GE1502)
% Professor: B. O'Connell
%%%%%%%%%%%%%%%%%%%%%%%%%%%%%%%%%%%%%%%%%%%%%%%%%%%%%%%%%%%%%%%%%%%%%%%%%%%%%%%%%%%%%%%%%%%%%%%%%%%%%%%%%%%%%%%%%%%%%%%%%%%%%%%%%%%%%%%%%%%%%%%%%%%%%%%%%%%%%%%%%%%

%%%%%%%%%%%%%%%%%%%%%%%%%%%%%%%%%%%%%%%%%%%%%%%%%%%%%%%%%%%%%%%%%%%%%%%%%%%%%%%%%%%%%%%%%%%%%%%%%%%%%%%%%%%%%%%%%%%%%%%%%%%%%%%%%%%%%%%%%%%%%%%%%%%%%%%%%%%%%%%%%%%
% Written By Michael Brodskiy
% Class: Cornerstone Engineering 1 & 2 (GE1501 & GE1502)
% Professor: B. O'Connell
%%%%%%%%%%%%%%%%%%%%%%%%%%%%%%%%%%%%%%%%%%%%%%%%%%%%%%%%%%%%%%%%%%%%%%%%%%%%%%%%%%%%%%%%%%%%%%%%%%%%%%%%%%%%%%%%%%%%%%%%%%%%%%%%%%%%%%%%%%%%%%%%%%%%%%%%%%%%%%%%%%%

\documentclass[12pt]{article} 
\usepackage{alphalph}
\usepackage[utf8]{inputenc}
\usepackage[russian,english]{babel}
\usepackage{titling}
\usepackage{amsmath}
\usepackage{graphicx}
\usepackage{enumitem}
\usepackage{amssymb}
\usepackage[super]{nth}
\usepackage{everysel}
\usepackage{ragged2e}
\usepackage{geometry}
\usepackage{multicol}
\usepackage{fancyhdr}
\usepackage{cancel}
\usepackage{siunitx}
\geometry{top=1.0in,bottom=1.0in,left=1.0in,right=1.0in}
\newcommand{\subtitle}[1]{%
  \posttitle{%
    \par\end{center}
    \begin{center}\large#1\end{center}
    \vskip0.5em}%

}
\usepackage{hyperref}
\hypersetup{
colorlinks=true,
linkcolor=blue,
filecolor=magenta,      
urlcolor=blue,
citecolor=blue,
}


\hypersetup{
colorlinks=true,
linkcolor=black,
filecolor=black,      
urlcolor=black,
citecolor=black,
}

\title{Project 2 Milestone 2 Flowchart}
\date{\today}
\author{Michael Brodskiy\\ \small Professor: B. O'Connell}

\begin{document}

\maketitle

\begin{figure}[H]
  \centering
  \tikzset{every picture/.style={line width=0.75pt}} %set default line width to 0.75pt        

\begin{tikzpicture}[x=0.75pt,y=0.75pt,yscale=-.55,xscale=.55]
%uncomment if require: \path (0,582); %set diagram left start at 0, and has height of 582

%Shape: Ellipse [id:dp3892963566590173] 
\draw   (11,41.5) .. controls (11,21.34) and (38.31,5) .. (72,5) .. controls (105.69,5) and (133,21.34) .. (133,41.5) .. controls (133,61.66) and (105.69,78) .. (72,78) .. controls (38.31,78) and (11,61.66) .. (11,41.5) -- cycle ;
%Straight Lines [id:da19887855056316295] 
\draw    (72,78) -- (72,125.42) ;
\draw [shift={(72,127.42)}, rotate = 270] [color={rgb, 255:red, 0; green, 0; blue, 0 }  ][line width=0.75]    (10.93,-3.29) .. controls (6.95,-1.4) and (3.31,-0.3) .. (0,0) .. controls (3.31,0.3) and (6.95,1.4) .. (10.93,3.29)   ;
%Shape: Rectangle [id:dp014208133529669098] 
\draw   (11,128) -- (133,128) -- (133,201) -- (11,201) -- cycle ;
%Flowchart: Document [id:dp8688522724734018] 
\draw   (11,486) -- (133,486) -- (133,546.23) .. controls (56.75,546.23) and (72,567.94) .. (11,553.89) -- cycle ;
%Straight Lines [id:da5839938576291159] 
\draw    (72,200.42) -- (72,247.84) ;
\draw [shift={(72,249.84)}, rotate = 270] [color={rgb, 255:red, 0; green, 0; blue, 0 }  ][line width=0.75]    (10.93,-3.29) .. controls (6.95,-1.4) and (3.31,-0.3) .. (0,0) .. controls (3.31,0.3) and (6.95,1.4) .. (10.93,3.29)   ;
%Shape: Rectangle [id:dp912979444091131] 
\draw   (183,484) -- (305,484) -- (305,557) -- (183,557) -- cycle ;
%Straight Lines [id:da5083600057856885] 
\draw    (244,484.5) -- (244,437.08) ;
\draw [shift={(244,435.08)}, rotate = 90] [color={rgb, 255:red, 0; green, 0; blue, 0 }  ][line width=0.75]    (10.93,-3.29) .. controls (6.95,-1.4) and (3.31,-0.3) .. (0,0) .. controls (3.31,0.3) and (6.95,1.4) .. (10.93,3.29)   ;
%Straight Lines [id:da1366751880696635] 
\draw    (133,520.23) -- (180.42,520.23) ;
\draw [shift={(182.42,520.23)}, rotate = 180] [color={rgb, 255:red, 0; green, 0; blue, 0 }  ][line width=0.75]    (10.93,-3.29) .. controls (6.95,-1.4) and (3.31,-0.3) .. (0,0) .. controls (3.31,0.3) and (6.95,1.4) .. (10.93,3.29)   ;
%Shape: Rectangle [id:dp1281567774491723] 
\draw   (183,362) -- (305,362) -- (305,435) -- (183,435) -- cycle ;
%Shape: Boxed Line [id:dp041257977100174203] 
\draw    (305,399) -- (352.42,399) ;
\draw [shift={(354.42,399)}, rotate = 180] [color={rgb, 255:red, 0; green, 0; blue, 0 }  ][line width=0.75]    (10.93,-3.29) .. controls (6.95,-1.4) and (3.31,-0.3) .. (0,0) .. controls (3.31,0.3) and (6.95,1.4) .. (10.93,3.29)   ;
%Flowchart: Decision [id:dp6100652602081542] 
\draw   (415.42,362.5) -- (476.42,399) -- (415.42,435.5) -- (354.42,399) -- cycle ;
%Straight Lines [id:da2210516933251061] 
\draw    (415.42,362.5) -- (415.42,315.08) ;
\draw [shift={(415.42,313.08)}, rotate = 90] [color={rgb, 255:red, 0; green, 0; blue, 0 }  ][line width=0.75]    (10.93,-3.29) .. controls (6.95,-1.4) and (3.31,-0.3) .. (0,0) .. controls (3.31,0.3) and (6.95,1.4) .. (10.93,3.29)   ;
%Flowchart: Decision [id:dp5795837488593116] 
\draw   (415.42,241) -- (476.42,277.5) -- (415.42,314) -- (354.42,277.5) -- cycle ;
%Straight Lines [id:da8276046312641692] 
\draw    (415.42,240.5) -- (415.42,193.08) ;
\draw [shift={(415.42,191.08)}, rotate = 90] [color={rgb, 255:red, 0; green, 0; blue, 0 }  ][line width=0.75]    (10.93,-3.29) .. controls (6.95,-1.4) and (3.31,-0.3) .. (0,0) .. controls (3.31,0.3) and (6.95,1.4) .. (10.93,3.29)   ;
%Shape: Rectangle [id:dp40043316067990364] 
\draw   (355,119) -- (477,119) -- (477,192) -- (355,192) -- cycle ;
%Straight Lines [id:da00836568891618028] 
\draw    (476.42,399) -- (523.84,399) ;
\draw [shift={(525.84,399)}, rotate = 180] [color={rgb, 255:red, 0; green, 0; blue, 0 }  ][line width=0.75]    (10.93,-3.29) .. controls (6.95,-1.4) and (3.31,-0.3) .. (0,0) .. controls (3.31,0.3) and (6.95,1.4) .. (10.93,3.29)   ;
%Flowchart: Document [id:dp27877003106125287] 
\draw   (11,251) -- (133,251) -- (133,311.23) .. controls (56.75,311.23) and (72,332.94) .. (11,318.89) -- cycle ;
%Straight Lines [id:da9872348183315431] 
\draw    (72,318.42) -- (72,365.84) ;
\draw [shift={(72,367.84)}, rotate = 270] [color={rgb, 255:red, 0; green, 0; blue, 0 }  ][line width=0.75]    (10.93,-3.29) .. controls (6.95,-1.4) and (3.31,-0.3) .. (0,0) .. controls (3.31,0.3) and (6.95,1.4) .. (10.93,3.29)   ;
%Flowchart: Document [id:dp8293742293432909] 
\draw   (11,369) -- (133,369) -- (133,429.23) .. controls (56.75,429.23) and (72,450.94) .. (11,436.89) -- cycle ;
%Straight Lines [id:da33583291117449665] 
\draw    (72,436.5) -- (72,483.92) ;
\draw [shift={(72,485.92)}, rotate = 270] [color={rgb, 255:red, 0; green, 0; blue, 0 }  ][line width=0.75]    (10.93,-3.29) .. controls (6.95,-1.4) and (3.31,-0.3) .. (0,0) .. controls (3.31,0.3) and (6.95,1.4) .. (10.93,3.29)   ;
%Flowchart: Document [id:dp5517234599271172] 
\draw   (183,129) -- (305,129) -- (305,189.23) .. controls (228.75,189.23) and (244,210.94) .. (183,196.89) -- cycle ;
%Straight Lines [id:da11462356345116653] 
\draw    (244,196.5) -- (244,243.92) ;
\draw [shift={(244,245.92)}, rotate = 270] [color={rgb, 255:red, 0; green, 0; blue, 0 }  ][line width=0.75]    (10.93,-3.29) .. controls (6.95,-1.4) and (3.31,-0.3) .. (0,0) .. controls (3.31,0.3) and (6.95,1.4) .. (10.93,3.29)   ;
%Straight Lines [id:da2650692735370479] 
\draw    (355,155.5) -- (307,155.5) ;
\draw [shift={(305,155.5)}, rotate = 360] [color={rgb, 255:red, 0; green, 0; blue, 0 }  ][line width=0.75]    (10.93,-3.29) .. controls (6.95,-1.4) and (3.31,-0.3) .. (0,0) .. controls (3.31,0.3) and (6.95,1.4) .. (10.93,3.29)   ;
%Shape: Rectangle [id:dp6811663085657764] 
\draw   (183,247) -- (305,247) -- (305,320) -- (183,320) -- cycle ;
%Straight Lines [id:da7744902079092426] 
\draw    (244,319.5) -- (244,359.92) ;
\draw [shift={(244,361.92)}, rotate = 270] [color={rgb, 255:red, 0; green, 0; blue, 0 }  ][line width=0.75]    (10.93,-3.29) .. controls (6.95,-1.4) and (3.31,-0.3) .. (0,0) .. controls (3.31,0.3) and (6.95,1.4) .. (10.93,3.29)   ;
%Shape: Rectangle [id:dp6868092748581316] 
\draw   (525,239) -- (647,239) -- (647,312) -- (525,312) -- cycle ;
%Flowchart: Decision [id:dp07121461880236302] 
\draw   (585.42,362.5) -- (646.42,399) -- (585.42,435.5) -- (524.42,399) -- cycle ;
%Straight Lines [id:da05515754453210531] 
\draw    (585.42,459.79) -- (246,459.79) ;
\draw [shift={(244,459.79)}, rotate = 360] [color={rgb, 255:red, 0; green, 0; blue, 0 }  ][line width=0.75]    (10.93,-3.29) .. controls (6.95,-1.4) and (3.31,-0.3) .. (0,0) .. controls (3.31,0.3) and (6.95,1.4) .. (10.93,3.29)   ;
%Straight Lines [id:da41299779400165915] 
\draw    (585.42,435.5) -- (585.42,459.79) ;
%Straight Lines [id:da27670933317397006] 
\draw    (585.42,362.5) -- (585.42,315.08) ;
\draw [shift={(585.42,313.08)}, rotate = 90] [color={rgb, 255:red, 0; green, 0; blue, 0 }  ][line width=0.75]    (10.93,-3.29) .. controls (6.95,-1.4) and (3.31,-0.3) .. (0,0) .. controls (3.31,0.3) and (6.95,1.4) .. (10.93,3.29)   ;
%Flowchart: Decision [id:dp4806907963744267] 
\draw   (586,117.08) -- (647,153.58) -- (586,190.08) -- (525,153.58) -- cycle ;
%Straight Lines [id:da09784740458097874] 
\draw    (586,117.5) -- (586,70.08) ;
\draw [shift={(586,68.08)}, rotate = 90] [color={rgb, 255:red, 0; green, 0; blue, 0 }  ][line width=0.75]    (10.93,-3.29) .. controls (6.95,-1.4) and (3.31,-0.3) .. (0,0) .. controls (3.31,0.3) and (6.95,1.4) .. (10.93,3.29)   ;
%Straight Lines [id:da6080018155245943] 
\draw    (586,239.5) -- (586,192.08) ;
\draw [shift={(586,190.08)}, rotate = 90] [color={rgb, 255:red, 0; green, 0; blue, 0 }  ][line width=0.75]    (10.93,-3.29) .. controls (6.95,-1.4) and (3.31,-0.3) .. (0,0) .. controls (3.31,0.3) and (6.95,1.4) .. (10.93,3.29)   ;
%Straight Lines [id:da4919984780577087] 
\draw    (498.84,277.5) -- (476.42,277.5) ;
%Flowchart: Document [id:dp09854064656975048] 
\draw   (525,1) -- (647,1) -- (647,61.23) .. controls (570.75,61.23) and (586,82.94) .. (525,68.89) -- cycle ;
%Straight Lines [id:da6227975102616017] 
\draw    (525,37.5) -- (477,37.5) ;
\draw [shift={(475,37.5)}, rotate = 360] [color={rgb, 255:red, 0; green, 0; blue, 0 }  ][line width=0.75]    (10.93,-3.29) .. controls (6.95,-1.4) and (3.31,-0.3) .. (0,0) .. controls (3.31,0.3) and (6.95,1.4) .. (10.93,3.29)   ;
%Flowchart: Terminator [id:dp07962655437406885] 
\draw   (372.52,1) -- (455.48,1) .. controls (466.26,1) and (475,17.34) .. (475,37.5) .. controls (475,57.66) and (466.26,74) .. (455.48,74) -- (372.52,74) .. controls (361.74,74) and (353,57.66) .. (353,37.5) .. controls (353,17.34) and (361.74,1) .. (372.52,1) -- cycle ;
%Straight Lines [id:da9765378738053909] 
\draw    (500,37.5) -- (498.84,277.5) ;

% Text Node
\draw (72,41.5) node  [font=\tiny] [align=left] {\begin{minipage}[lt]{60.27pt}\setlength\topsep{0pt}
\begin{center}
{\fontfamily{pcr}\selectfont Start}\\Simple Encryption\\Game
\end{center}

\end{minipage}};
% Text Node
\draw (72,164.5) node  [font=\tiny] [align=left] {\begin{minipage}[lt]{47.58pt}\setlength\topsep{0pt}
\begin{center}
2+ students\\play the game
\end{center}

\end{minipage}};
% Text Node
\draw (72,522.5) node  [font=\tiny] [align=left] {\begin{minipage}[lt]{45.59pt}\setlength\topsep{0pt}
\begin{center}
Student one\\inputs phrase
\end{center}

\end{minipage}};
% Text Node
\draw (244,520.5) node  [font=\tiny] [align=left] {\begin{minipage}[lt]{61.59pt}\setlength\topsep{0pt}
\begin{center}
Record phrase,\\difficulty, standard,\\ages in database
\end{center}

\end{minipage}};
% Text Node
\draw (244,398.5) node  [font=\tiny] [align=left] {\begin{minipage}[lt]{55.51pt}\setlength\topsep{0pt}
\begin{center}
Student two\\guesses letter or\\phrase/standard
\end{center}

\end{minipage}};
% Text Node
\draw (416.42,399) node  [font=\tiny] [align=left] {\begin{minipage}[lt]{33.69pt}\setlength\topsep{0pt}
\begin{center}
Letter\\guessed?
\end{center}

\end{minipage}};
% Text Node
\draw (417.42,337.79) node [anchor=west] [inner sep=0.75pt]  [font=\tiny] [align=left] {no};
% Text Node
\draw (416.42,282.5) node  [font=\tiny] [align=left] {\begin{minipage}[lt]{57.1pt}\setlength\topsep{0pt}
\begin{center}
Phrase/standard \\correct?
\end{center}

\end{minipage}};
% Text Node
\draw (417.42,215.79) node [anchor=west] [inner sep=0.75pt]  [font=\tiny] [align=left] {no};
% Text Node
\draw (416,155.5) node  [font=\tiny] [align=left] {\begin{minipage}[lt]{67.42pt}\setlength\topsep{0pt}
\begin{center}
Phrase re-encrypted\\with different \\standard
\end{center}

\end{minipage}};
% Text Node
\draw (501.13,396) node [anchor=south] [inner sep=0.75pt]  [font=\tiny] [align=left] {yes};
% Text Node
\draw (72,287.5) node  [font=\tiny] [align=left] {\begin{minipage}[lt]{58.69pt}\setlength\topsep{0pt}
\begin{center}
Players enter age
\end{center}

\end{minipage}};
% Text Node
\draw (72,405.5) node  [font=\tiny] [align=left] {\begin{minipage}[lt]{57.49pt}\setlength\topsep{0pt}
\begin{center}
Student one\\enters encryption\\standard
\end{center}

\end{minipage}};
% Text Node
\draw (244,165.5) node  [font=\tiny] [align=left] {\begin{minipage}[lt]{57.49pt}\setlength\topsep{0pt}
\begin{center}
Student one\\enters encryption\\standard
\end{center}

\end{minipage}};
% Text Node
\draw (244,283.5) node  [font=\tiny] [align=left] {\begin{minipage}[lt]{40.82pt}\setlength\topsep{0pt}
\begin{center}
Record new\\standard in\\database
\end{center}

\end{minipage}};
% Text Node
\draw (586,275.5) node  [font=\tiny] [align=left] {\begin{minipage}[lt]{47.97pt}\setlength\topsep{0pt}
\begin{center}
Reveal\\corresponding\\letters
\end{center}

\end{minipage}};
% Text Node
\draw (586.42,399) node  [font=\tiny] [align=left] {\begin{minipage}[lt]{36.07pt}\setlength\topsep{0pt}
\begin{center}
Letter\\in phrase?
\end{center}

\end{minipage}};
% Text Node
\draw (587.42,447.64) node [anchor=west] [inner sep=0.75pt]  [font=\tiny] [align=left] {no};
% Text Node
\draw (587.42,337.79) node [anchor=west] [inner sep=0.75pt]  [font=\tiny] [align=left] {yes};
% Text Node
\draw (586,153.58) node  [font=\tiny] [align=left] {\begin{minipage}[lt]{33.69pt}\setlength\topsep{0pt}
\begin{center}
All letters\\guessed?
\end{center}

\end{minipage}};
% Text Node
\draw (588,92.79) node [anchor=west] [inner sep=0.75pt]  [font=\tiny] [align=left] {yes};
% Text Node
\draw (486.13,274.5) node [anchor=south] [inner sep=0.75pt]  [font=\tiny] [align=left] {yes};
% Text Node
\draw (586,37.5) node  [font=\tiny] [align=left] {\begin{minipage}[lt]{51.55pt}\setlength\topsep{0pt}
\begin{center}
Student two\\inputs standard\\guess
\end{center}

\end{minipage}};
% Text Node
\draw (414,37.5) node  [font=\tiny] [align=left] {\begin{minipage}[lt]{42.4pt}\setlength\topsep{0pt}
\begin{center}
Game ends,\\statistics are\\displayed
\end{center}

\end{minipage}};


\end{tikzpicture}

  \caption{A depiction of the simple encryption game. The encryption standard (to be simple and guessable) will be plus or minus-based for easy/medium modes, and multiplication/division based for hard mode. For data purposes, the standard will be recorded as $<$\textsc{operation}$><$\textsc{integer}$>$; for example a standard of a character times 5 would be *5. An optional hint can be displayed if stuck that lets player two know which operation is being used. The exhibit itself will have more information, with text and images, describing the history of encryption and modern counterparts, and possible other interactive elements (some ideas are being developed). An alternate single player version could have a database of phrases, with difficulty corresponding to phrase length and attempts to guess until failure.}
  \label{fig:1}
\end{figure}

\end{document}

