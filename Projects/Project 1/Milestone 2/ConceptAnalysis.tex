%%%%%%%%%%%%%%%%%%%%%%%%%%%%%%%%%%%%%%%%%%%%%%%%%%%%%%%%%%%%%%%%%%%%%%%%%%%%%%%%%%%%%%%%%%%%%%%%%%%%%%%%%%%%%%%%%%%%%%%%%%%%%%%%%%%%%%%%%%%%%%%%%%%%%%%%%%%%%%%%%%%
% Written By Michael Brodskiy
% Class: Cornerstone Engineering 1 & 2 (GE1501 & GE1502)
% Professor: B. O'Connell
%%%%%%%%%%%%%%%%%%%%%%%%%%%%%%%%%%%%%%%%%%%%%%%%%%%%%%%%%%%%%%%%%%%%%%%%%%%%%%%%%%%%%%%%%%%%%%%%%%%%%%%%%%%%%%%%%%%%%%%%%%%%%%%%%%%%%%%%%%%%%%%%%%%%%%%%%%%%%%%%%%%

%%%%%%%%%%%%%%%%%%%%%%%%%%%%%%%%%%%%%%%%%%%%%%%%%%%%%%%%%%%%%%%%%%%%%%%%%%%%%%%%%%%%%%%%%%%%%%%%%%%%%%%%%%%%%%%%%%%%%%%%%%%%%%%%%%%%%%%%%%%%%%%%%%%%%%%%%%%%%%%%%%%
% Written By Michael Brodskiy
% Class: Cornerstone Engineering 1 & 2 (GE1501 & GE1502)
% Professor: B. O'Connell
%%%%%%%%%%%%%%%%%%%%%%%%%%%%%%%%%%%%%%%%%%%%%%%%%%%%%%%%%%%%%%%%%%%%%%%%%%%%%%%%%%%%%%%%%%%%%%%%%%%%%%%%%%%%%%%%%%%%%%%%%%%%%%%%%%%%%%%%%%%%%%%%%%%%%%%%%%%%%%%%%%%

\documentclass[12pt]{article} 
\usepackage{alphalph}
\usepackage[utf8]{inputenc}
\usepackage[russian,english]{babel}
\usepackage{titling}
\usepackage{amsmath}
\usepackage{graphicx}
\usepackage{enumitem}
\usepackage{amssymb}
\usepackage[super]{nth}
\usepackage{everysel}
\usepackage{ragged2e}
\usepackage{geometry}
\usepackage{multicol}
\usepackage{fancyhdr}
\usepackage{cancel}
\usepackage{siunitx}
\geometry{top=1.0in,bottom=1.0in,left=1.0in,right=1.0in}
\newcommand{\subtitle}[1]{%
  \posttitle{%
    \par\end{center}
    \begin{center}\large#1\end{center}
    \vskip0.5em}%

}
\usepackage{hyperref}
\hypersetup{
colorlinks=true,
linkcolor=blue,
filecolor=magenta,      
urlcolor=blue,
citecolor=blue,
}


\title{Concept and Priority Analysis}
\date{\today}
\author{Michael Brodskiy\\ \small Professor: B. O'Connell}

\begin{document}

\maketitle

\begin{center}
  \underline{Priorities (in order of importance):}
\end{center}

\begin{enumerate}

  \item Feasibility (whether the design can realistically be achieved)

    \begin{itemize}

      \item A rating of one means the design is highly unlikely to work, a rating of ten means the design is highly likely to work

    \end{itemize}

  \item Accuracy (ability to launch the projectile to a consistent spot)

    \begin{itemize}

      \item A rating of one means the device is highly unlikely to hit targets consistently, while a rating of ten means the device is highly likely to fire accurately

    \end{itemize}

  \item Simplicity (limited amount of moving parts)

    \begin{itemize}

      \item A rating of one means the design has too many moving parts, decreasing its ease of construction and use, while a rating of ten means the design has little to no moving parts

    \end{itemize}

  \item Durability (ability of the device to withstand repeated usage)

    \begin{itemize}

      \item A rating of one means the device is unlikely to resist damage with use or operation, while a rating of ten means the device is likely to be resilient to operation

    \end{itemize}

\end{enumerate}

\begin{center}
  Crossbow Wrist Launcher\footnote{Two crossbow designs were proposed — one has been omitted to avoid repetition, leaving seven designs total}
\end{center}

\begin{enumerate}

  \item Feasibility — 7 (technology is proven to work, but reloading seems difficult)

    \begin{itemize}

      \item Pro: Because the main component of the design is based on existing technology, the feasibility is proven

      \item Con: The reloading mechanism that releases the ping pong ball from a cup seems overly complicated and not likely to work

    \end{itemize}

  \item Accuracy — 9 (Power output makes it accuracy, loses one point because it is questionable whether the accuracy suits our needs)

    \begin{itemize}

      \item Pro: The power output of the crossbow makes it fairly accurate, as long as it is well-calibrated

      \item Con: If the goal is to make it into a cup, there will be fairly little arch to the projectile's motion, making success unlikely

    \end{itemize}

  \item Simplicity — 5 (The complex reloading mechanism, coupled with the simple crossbow, make it neither simple nor complex overall)

    \begin{itemize}

      \item Pro: The wrist strap concept makes this design fairly simple to transport, in addition to being lightweight

      \item Con: There are a lot of moving parts in the reloading mechanism which can cause an increase in wear and tear

    \end{itemize}

  \item Durability — 6 (There is nothing that suggests the crossbow will be highly or barely durable)

    \begin{itemize}

      \item Pro: The crossbow itself should be fairly strong and reliable

      \item Con: The complexity of the reloading mechanism increases the chances of it breaking

    \end{itemize}

\end{enumerate}

\begin{center}
  Mangonellian
\end{center}

\begin{enumerate}

  \item Feasibility — 8 (The overall design is fairly easy to construct, but the wind up mechanism may be a bit difficult)

    \begin{itemize}

      \item Pro: The mangonel, similar to the crossbow, is a tested, historically-verified projectile-launching machine, which compliments its feasibility 

      \item Con: The wind-up mechanism might be difficult to implement for a single-handed individual

    \end{itemize}

  \item Accuracy — 6 (The possibility of leading left or right upon launch makes the device not too accurate laterally)

    \begin{itemize}

      \item Pro: As a result of the launching mechanism, the distance the projectile is launched should stay fairly consistent

      \item Con: Depending on the stability of the projectile in the basket and the device itself, the projectile may lead left or right on launch

    \end{itemize}

  \item Simplicity — 7 (The design doesn't depend on too many components, with the exception of the wind-up)

    \begin{itemize}

      \item Pro: There aren't too many components, making construction and operation fairly easy

      \item Con: The design seems a bit bulky and unlikely to fit size requirements

    \end{itemize}

  \item Durability — 8 (Similar devices generally utilize strong materials which allow for easy reuse, some points lost because of the dependence on a string for winding)

    \begin{itemize}

      \item Pro: The sturdy base and overall strong material that should accompany this design make it fairly reliable

      \item Con: The winding mechanism could wear the string quickly, causing it too snap (and possibly making operation of the device more dangerous)

    \end{itemize}

\end{enumerate}

\begin{center}
  Concentric Tube Launcher
\end{center}

\begin{enumerate}

  \item Feasibility — 3 (The complexity of the design make it unlikely to function as intended)

    \begin{itemize}

      \item Pro: In general, the gun-like construction of the launcher makes it a good concept

      \item Con: The complexity of reloading may be unnecessarily difficult and hard to construct

    \end{itemize}

  \item Accuracy — 4 (Dependence on elastics make this device unlikely to fire as intended or consistently)

    \begin{itemize}

      \item Pro: If the barrel is constructed with a good radius to hold the projectile, the design should be quite accurate

      \item Con: The reliance on rubber bands and difficulty to construct and ideal barrel make it unlikely to fire consistently

    \end{itemize}

  \item Simplicity — 2 (The idea of having to utilize two tubes and pressing one in to reload adds a lot of part movement that makes the device complex)

    \begin{itemize}

      \item Pro: The reloading mechanism is probably as simple as can be: simply pop the projectile into the barrel

      \item Con: Reloading depends on a lot of undependable moving parts, making the design quite complex

    \end{itemize}

  \item Durability — 6 (In general, the device should be fairly stable, but the reliance on elastics causes a drop-off in effectiveness)

    \begin{itemize}

      \item Pro: Although made of very light material, the method by which the launcher fires makes it highly unlikely to be damaged during operation

      \item Con: Rubber bands can dry out fairly quickly, decreasing the effectiveness of the launcher over time

    \end{itemize}

\end{enumerate}

\begin{center}
  Blowgun Launcher
\end{center}

\begin{enumerate}

  \item Feasibility — 8 (The device is quite easy to operate and construct, making it easy to test and employ)

    \begin{itemize}

      \item Pro: The design is fairly easy to construct, relies on very little materials, and is thoroughly tested

      \item Con: Its effectiveness is difficult to determine until it is fully constructed

    \end{itemize}

  \item Accuracy — 6 (Because of the dependence on breathing output, the accuracy of the device has a great amount of variability)

    \begin{itemize}

      \item Pro: The simple, single-tube design makes the launcher fairly consist in its ability to fire the projectile

      \item Con: The accuracy also depends on the user's ability to blow into the tube

    \end{itemize}

  \item Simplicity — 10 (No moving parts make the device nearly impossible to malfunction)

    \begin{itemize}

      \item Pro: The design is probably most simple of any provided, and can be constructed fairly easily

      \item Con: The reloading mechanism seems unnecessarily complicated; would probably be better to go with something less complex

    \end{itemize}

  \item Durability — 9 (In terms of purely operation, the device should take very little to no damage each use)

    \begin{itemize}

      \item Pro: The lack of any moving parts in the actual launching mechanism make the launcher highly unlikely to break

      \item Con: The structural integrity of the launcher itself is easily compromised by outside forces

    \end{itemize}

\end{enumerate}

\begin{center}
  Step-on Catapult
\end{center}

\begin{enumerate}

  \item Feasibility — 7 (As history has shown, catapults generally work, but the stepping component adds a layer of complexity)

    \begin{itemize}

      \item Pro: The fairly simplistic idea make the design easy to model and prototype, as well as test in real conditions

      \item Con: The material will need to be fairly resilient to resist collapsing under stress from stepping

    \end{itemize}

  \item Accuracy — 4 (The reliance on a stepping motion make consistent firing unlikely)

    \begin{itemize}

      \item Pro: The range of launching the object, similar to the mangonel, should stay fairly consistent

      \item Con: The reliance on someone stepping adds a lot of factors that could influence the output of the projectile

    \end{itemize}

  \item Simplicity — 8 (The design and concept overall relies on very little moving parts)

    \begin{itemize}

      \item Pro: The concept itself is fairly simple and, in theory, should work fairly easily

      \item Con: It is, however, difficult to imagine the device operating consistently given the requirement of stepping

    \end{itemize}

  \item Durability — 4 (It is difficult to imagine the device holding up to constant stepping)

    \begin{itemize}

      \item Pro: Most likely, strong material will be used to construct the design

      \item Con: Repeated stepping may cause the device to collapse or be damaged in some way

    \end{itemize}

\end{enumerate}

\begin{center}
  One-Handed Catapult
\end{center}

\begin{enumerate}

  \item Feasibility — 8 (In general, the device seems like it can operate with very little difficulty)

    \begin{itemize}

      \item Pro: It is definitely possible to construct this and operate it with one hand

      \item Con: It is difficult to imagine what kind of material we could use to bend the catapult backwards and release

    \end{itemize}

  \item Accuracy — 5 (The varying ability does allow for some improved accuracy, but this devices seems likely to lead left or right)

    \begin{itemize}

      \item Pro: The ability to vary the amount the catapult is pulled back allows for improved and more adjustable range on the projectile

      \item Con: Again, similar to the mangonel and step-on catapult, the left-right variation in launching a projectile may be difficult to prevent

    \end{itemize}

  \item Simplicity — 8 (The device doesn't depend on too many parts, and, as such, should be fairly stable)

    \begin{itemize}

      \item Pro: The design is quite simple, and has fairly few moving parts

      \item Con: The material used for the stalk of the crossbow seems a bit complex, it is unclear what could be used

    \end{itemize}

  \item Durability — 6 (Nothing indicates whether the device should be of above or below average strength)

    \begin{itemize}

      \item Pro: The sturdy base and general design of a catapult should be fairly strong and, therefore, resilient

      \item Con: Again, it is unclear whether the material for the stalk itself would be durable

    \end{itemize}

\end{enumerate}

\begin{center}
  Clip-On Slingshot
\end{center}

\begin{enumerate}

  \item Feasibility — 7 (The device seems fairly easy to construct and test, but the clipping aspect needs some more work and explanation)

    \begin{itemize}

      \item Pro: A slingshot is, as is well known, quite reliable for launching projectiles, and therefore, should work well

      \item Con: The clipping aspect of the device needs to be quite sturdy, otherwise the device may rotate or unclip during operation

    \end{itemize}

  \item Accuracy — 7 (The ability to aim in more directions gives more freedom, but makes it slightly less accurate)

    \begin{itemize}

      \item Pro: The slingshot should be able to consistently output a good amount of power

      \item Con: The fact that it is necessary to aim side to side, as well as up and down, make it a bit difficult to score into a cup

    \end{itemize}

  \item Simplicity — 9 (Overall, the design depends on very little moving parts, with the exception of elastics)

    \begin{itemize}

      \item Pro: The design is definitely quite simple, and it only depends on the elastics for moving parts

      \item Con: It is unclear how to construct durable clips with easily accessible materials, in addition to making these clips adjustable for different surfaces

    \end{itemize}

  \item Durability — 6 (The device itself is fairly sturdy, but the reliance on elastics allows for a lot of drop off in effectiveness)

    \begin{itemize}

      \item Pro: The base itself and the clips should be fairly sturdy

      \item Con: The elastics, especially with age and use, will dry out, and most likely cause a decline in effectiveness

    \end{itemize}

\end{enumerate}

\begin{center}
  \underline{Materials} \\
  \vspace{10pt}
  \begin{tabular}{|c|}
    \hline
    Cardboard\\
    \hline
    Tape\\
    \hline
    Tube\footnote{Possibly from toilet paper}\\
    \hline
    ABS or PLA\footnote{For 3D printing}\\
    \hline
    Rubber bands\\
    \hline
    String\\
    \hline
  \end{tabular}

\end{center}

\end{document}

