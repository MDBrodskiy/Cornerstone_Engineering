%%%%%%%%%%%%%%%%%%%%%%%%%%%%%%%%%%%%%%%%%%%%%%%%%%%%%%%%%%%%%%%%%%%%%%%%%%%%%%%%%%%%%%%%%%%%%%%%%%%%%%%%%%%%%%%%%%%%%%%%%%%%%%%%%%%%%%%%%%%%%%%%%%%%%%%%%%%%%%%%%%%
% Written By Michael Brodskiy
% Class: Cornerstone Engineering 1 & 2 (GE1501 & GE1502)
% Professor: B. O'Connell
%%%%%%%%%%%%%%%%%%%%%%%%%%%%%%%%%%%%%%%%%%%%%%%%%%%%%%%%%%%%%%%%%%%%%%%%%%%%%%%%%%%%%%%%%%%%%%%%%%%%%%%%%%%%%%%%%%%%%%%%%%%%%%%%%%%%%%%%%%%%%%%%%%%%%%%%%%%%%%%%%%%

%%%%%%%%%%%%%%%%%%%%%%%%%%%%%%%%%%%%%%%%%%%%%%%%%%%%%%%%%%%%%%%%%%%%%%%%%%%%%%%%%%%%%%%%%%%%%%%%%%%%%%%%%%%%%%%%%%%%%%%%%%%%%%%%%%%%%%%%%%%%%%%%%%%%%%%%%%%%%%%%%%%
% Written By Michael Brodskiy
% Class: Cornerstone Engineering 1 & 2 (GE1501 & GE1502)
% Professor: B. O'Connell
%%%%%%%%%%%%%%%%%%%%%%%%%%%%%%%%%%%%%%%%%%%%%%%%%%%%%%%%%%%%%%%%%%%%%%%%%%%%%%%%%%%%%%%%%%%%%%%%%%%%%%%%%%%%%%%%%%%%%%%%%%%%%%%%%%%%%%%%%%%%%%%%%%%%%%%%%%%%%%%%%%%

\documentclass[12pt]{article} 
\usepackage{alphalph}
\usepackage[utf8]{inputenc}
\usepackage[russian,english]{babel}
\usepackage{titling}
\usepackage{amsmath}
\usepackage{graphicx}
\usepackage{enumitem}
\usepackage{amssymb}
\usepackage[super]{nth}
\usepackage{everysel}
\usepackage{ragged2e}
\usepackage{geometry}
\usepackage{multicol}
\usepackage{fancyhdr}
\usepackage{cancel}
\usepackage{siunitx}
\geometry{top=1.0in,bottom=1.0in,left=1.0in,right=1.0in}
\newcommand{\subtitle}[1]{%
  \posttitle{%
    \par\end{center}
    \begin{center}\large#1\end{center}
    \vskip0.5em}%

}
\usepackage{hyperref}
\hypersetup{
colorlinks=true,
linkcolor=blue,
filecolor=magenta,      
urlcolor=blue,
citecolor=blue,
}


\title{The Blow Launcher}
\date{\today}
\author{Michael Brodskiy\\ \small Professor: B. O'Connell}

\begin{document}

\maketitle

\begin{figure}[h!]
  \centering
  \tikzset{every picture/.style={line width=0.75pt}} %set default line width to 0.75pt        

\begin{tikzpicture}[x=0.75pt,y=0.75pt,yscale=-1,xscale=1]
%uncomment if require: \path (0,375); %set diagram left start at 0, and has height of 375

%Flowchart: Stored Data [id:dp7020854973393003] 
\draw   (166.88,189) -- (245,189) .. controls (236.78,189) and (230.12,202.21) .. (230.12,218.5) .. controls (230.12,234.79) and (236.78,248) .. (245,248) -- (166.88,248) .. controls (158.66,248) and (152,234.79) .. (152,218.5) .. controls (152,202.21) and (158.66,189) .. (166.88,189) -- cycle ;
%Flowchart: Stored Data [id:dp17745579227095631] 
\draw   (158.88,248) -- (237,248) .. controls (228.78,248) and (222.12,261.21) .. (222.12,277.5) .. controls (222.12,293.79) and (228.78,307) .. (237,307) -- (158.88,307) .. controls (150.66,307) and (144,293.79) .. (144,277.5) .. controls (144,261.21) and (150.66,248) .. (158.88,248) -- cycle ;
%Flowchart: Terminator [id:dp12987464792265335] 
\draw   (161.29,307) -- (246.06,307) .. controls (257.07,307) and (266,319.98) .. (266,336) .. controls (266,352.02) and (257.07,365) .. (246.06,365) -- (161.29,365) .. controls (150.28,365) and (141.35,352.02) .. (141.35,336) .. controls (141.35,319.98) and (150.28,307) .. (161.29,307) -- cycle ;
%Shape: Boxed Line [id:dp02347467253997526] 
\draw    (386.42,189) -- (245,189) ;
%Shape: Boxed Line [id:dp07914747281461065] 
\draw    (527.84,189) -- (386.42,189) ;
%Shape: Arc [id:dp5416882594805525] 
\draw  [draw opacity=0] (527.84,189) .. controls (527.84,189) and (527.84,189) .. (527.84,189) .. controls (511.27,189) and (497.84,175.57) .. (497.84,159) .. controls (497.84,142.43) and (511.27,129) .. (527.84,129) .. controls (527.84,129) and (527.84,129) .. (527.84,129) -- (527.84,159) -- cycle ; \draw   (527.84,189) .. controls (527.84,189) and (527.84,189) .. (527.84,189) .. controls (511.27,189) and (497.84,175.57) .. (497.84,159) .. controls (497.84,142.43) and (511.27,129) .. (527.84,129) .. controls (527.84,129) and (527.84,129) .. (527.84,129) ;  
%Shape: Arc [id:dp5474126979604366] 
\draw  [draw opacity=0] (527.84,129) .. controls (527.84,129) and (527.84,129) .. (527.84,129) .. controls (544.41,129) and (557.84,142.43) .. (557.84,159) .. controls (557.84,175.57) and (544.41,189) .. (527.84,189) .. controls (527.84,189) and (527.84,189) .. (527.84,189) -- (527.84,159) -- cycle ; \draw   (527.84,129) .. controls (527.84,129) and (527.84,129) .. (527.84,129) .. controls (544.41,129) and (557.84,142.43) .. (557.84,159) .. controls (557.84,175.57) and (544.41,189) .. (527.84,189) .. controls (527.84,189) and (527.84,189) .. (527.84,189) ;  
%Shape: Boxed Line [id:dp7312749797342681] 
\draw    (527.84,129) -- (386.42,129) ;
%Shape: Boxed Line [id:dp6106897593312044] 
\draw    (386.42,129) -- (257.44,129) ;
%Shape: Arc [id:dp4419482433997406] 
\draw  [draw opacity=0] (257.44,129) .. controls (257.44,129) and (257.44,129) .. (257.44,129) .. controls (240.87,129) and (227.44,115.57) .. (227.44,99) -- (257.44,99) -- cycle ; \draw   (257.44,129) .. controls (257.44,129) and (257.44,129) .. (257.44,129) .. controls (240.87,129) and (227.44,115.57) .. (227.44,99) ;  
%Shape: Arc [id:dp9757784820393378] 
\draw  [draw opacity=0] (158.88,129) .. controls (175.45,129) and (188.88,115.57) .. (188.88,99) -- (158.88,99) -- cycle ; \draw   (158.88,129) .. controls (175.45,129) and (188.88,115.57) .. (188.88,99) ;  
%Shape: Diagonal Stripe [id:dp8622279132894688] 
\draw  [fill={rgb, 255:red, 0; green, 0; blue, 0 }  ,fill opacity=1 ] (286.14,104.01) -- (316.56,121.32) -- (182.32,125.49) -- (219.02,106.1) -- cycle ;
%Shape: Boxed Line [id:dp8393902716199675] 
\draw    (158.88,129) -- (103.58,130) ;
%Shape: Arc [id:dp8477854245084788] 
\draw  [draw opacity=0] (103.58,129) .. controls (120.15,129) and (133.58,142.43) .. (133.58,159) .. controls (133.58,175.57) and (120.15,189) .. (103.58,189) -- (103.58,159) -- cycle ; \draw   (103.58,129) .. controls (120.15,129) and (133.58,142.43) .. (133.58,159) .. controls (133.58,175.57) and (120.15,189) .. (103.58,189) ;  
%Shape: Boxed Line [id:dp3622748694370084] 
\draw    (245,189) -- (103.58,189) ;
%Straight Lines [id:da7926240509197231] 
\draw [color={rgb, 255:red, 255; green, 255; blue, 255 }  ,draw opacity=1 ][fill={rgb, 255:red, 255; green, 255; blue, 255 }  ,fill opacity=1 ]   (158.88,248) -- (245,248) ;
%Straight Lines [id:da2916687030607792] 
\draw [color={rgb, 255:red, 255; green, 255; blue, 255 }  ,draw opacity=1 ][fill={rgb, 255:red, 255; green, 255; blue, 255 }  ,fill opacity=1 ]   (158.88,248) -- (245,248) ;
%Straight Lines [id:da13038003848282242] 
\draw [color={rgb, 255:red, 255; green, 255; blue, 255 }  ,draw opacity=1 ][fill={rgb, 255:red, 255; green, 255; blue, 255 }  ,fill opacity=1 ]   (150.88,307) -- (237,307) ;
%Straight Lines [id:da7000012227824575] 
\draw [color={rgb, 255:red, 255; green, 255; blue, 255 }  ,draw opacity=1 ][fill={rgb, 255:red, 255; green, 255; blue, 255 }  ,fill opacity=1 ]   (158.88,307) -- (245,307) ;
%Flowchart: Terminator [id:dp8667157151263947] 
\draw   (189.48,2) -- (225.52,2) .. controls (230.2,2) and (234,23.83) .. (234,50.75) .. controls (234,77.67) and (230.2,99.5) .. (225.52,99.5) -- (189.48,99.5) .. controls (184.8,99.5) and (181,77.67) .. (181,50.75) .. controls (181,23.83) and (184.8,2) .. (189.48,2) -- cycle ;
%Straight Lines [id:da7095055647762649] 
\draw [color={rgb, 255:red, 255; green, 255; blue, 255 }  ,draw opacity=1 ][fill={rgb, 255:red, 255; green, 255; blue, 255 }  ,fill opacity=1 ]   (189.48,99.5) -- (227.44,99) ;
%Shape: Circle [id:dp006657138719728506] 
\draw   (191.88,20) .. controls (191.88,11.16) and (199.04,4) .. (207.88,4) .. controls (216.72,4) and (223.88,11.16) .. (223.88,20) .. controls (223.88,28.84) and (216.72,36) .. (207.88,36) .. controls (199.04,36) and (191.88,28.84) .. (191.88,20) -- cycle ;
%Shape: Circle [id:dp6645484876518424] 
\draw   (191.88,56) .. controls (191.88,47.16) and (199.04,40) .. (207.88,40) .. controls (216.72,40) and (223.88,47.16) .. (223.88,56) .. controls (223.88,64.84) and (216.72,72) .. (207.88,72) .. controls (199.04,72) and (191.88,64.84) .. (191.88,56) -- cycle ;
%Shape: Circle [id:dp2974845468289886] 
\draw   (191.88,91) .. controls (191.88,82.16) and (199.04,75) .. (207.88,75) .. controls (216.72,75) and (223.88,82.16) .. (223.88,91) .. controls (223.88,99.84) and (216.72,107) .. (207.88,107) .. controls (199.04,107) and (191.88,99.84) .. (191.88,91) -- cycle ;
%Straight Lines [id:da44337627056843143] 
\draw  [dash pattern={on 4.5pt off 4.5pt}]  (129.58,143) -- (501,143.5) ;
%Shape: Donut [id:dp5639385250613416] 
\draw  [fill={rgb, 255:red, 0; green, 0; blue, 0 }  ,fill opacity=1 ,even odd rule] (512,159) .. controls (512,150.25) and (519.09,143.16) .. (527.84,143.16) .. controls (536.59,143.16) and (543.69,150.25) .. (543.69,159) .. controls (543.69,167.75) and (536.59,174.84) .. (527.84,174.84) .. controls (519.09,174.84) and (512,167.75) .. (512,159)(497.84,159) .. controls (497.84,142.43) and (511.27,129) .. (527.84,129) .. controls (544.41,129) and (557.84,142.43) .. (557.84,159) .. controls (557.84,175.57) and (544.41,189) .. (527.84,189) .. controls (511.27,189) and (497.84,175.57) .. (497.84,159) ;
%Shape: Boxed Line [id:dp41615350387263916] 
\draw    (527.84,143.16) -- (501,143.5) ;
%Shape: Rectangle [id:dp7644990864790011] 
\draw  [color={rgb, 255:red, 255; green, 255; blue, 255 }  ,draw opacity=1 ][fill={rgb, 255:red, 255; green, 255; blue, 255 }  ,fill opacity=1 ] (175,129.5) -- (245,129.5) -- (245,169.5) -- (175,169.5) -- cycle ;
%Straight Lines [id:da8405108949064521] 
\draw  [dash pattern={on 4.5pt off 4.5pt}]  (129.58,174.69) -- (501,175.19) ;
%Shape: Boxed Line [id:dp6819604809628732] 
\draw    (527.84,174.84) -- (501,175.19) ;
%Straight Lines [id:da9497322294650778] 
\draw [line width=2.25]    (338,283) -- (248.74,249.41) ;
\draw [shift={(245,248)}, rotate = 20.62] [color={rgb, 255:red, 0; green, 0; blue, 0 }  ][line width=2.25]    (17.49,-5.26) .. controls (11.12,-2.23) and (5.29,-0.48) .. (0,0) .. controls (5.29,0.48) and (11.12,2.23) .. (17.49,5.26)   ;
%Straight Lines [id:da68271392613904] 
\draw [line width=2.25]    (129,88) -- (179.05,123.19) ;
\draw [shift={(182.32,125.49)}, rotate = 215.12] [color={rgb, 255:red, 0; green, 0; blue, 0 }  ][line width=2.25]    (17.49,-5.26) .. controls (11.12,-2.23) and (5.29,-0.48) .. (0,0) .. controls (5.29,0.48) and (11.12,2.23) .. (17.49,5.26)   ;
%Straight Lines [id:da4765070016353259] 
\draw [line width=2.25]    (77,206) -- (101.61,162.48) ;
\draw [shift={(103.58,159)}, rotate = 119.49] [color={rgb, 255:red, 0; green, 0; blue, 0 }  ][line width=2.25]    (17.49,-5.26) .. controls (11.12,-2.23) and (5.29,-0.48) .. (0,0) .. controls (5.29,0.48) and (11.12,2.23) .. (17.49,5.26)   ;

% Text Node
\draw (340,286) node [anchor=north west][inner sep=0.75pt]   [align=left] {1. Hold blowgun at\\handle};
% Text Node
\draw (127,85) node [anchor=south east] [inner sep=0.75pt]   [align=left] {2. Use thumb to\\release a ball};
% Text Node
\draw (77,209) node [anchor=north] [inner sep=0.75pt]   [align=left] {3. Blow into opening \\to launch ball};


\end{tikzpicture}

  \caption{Blow Launcher Concept Sketch}
  \label{fig:1}
\end{figure}

\paragraph{} The blow launcher is quite simple by design, allowing it to be made from a variety of materials, and to be quite light. The user holds the blowgun at the handle and uses their thumb to reload. Once reloaded, the user blows into the opening, launching the ball. The simplicity of the design makes it easily reproducible as well.

\newline
\vspace{25pt}

Source:

[1] Guthrie, “Projectile feeder for paint ball blowgun, has indexed loading disk having multiple yieldingly locked index positions and is mounted at outlet of housing,” Jan. 04, 2002 Accessed: Sep. 24, 2022. [Online]. Available: \href{https://www-webofscience-com.ezproxy.neu.edu/wos/alldb/full-record/DIIDW:2002225069}{https://www-webofscience-com.ezproxy.neu.ed\\u/wos/alldb/full-record/DIIDW:2002225069}

\end{document}

