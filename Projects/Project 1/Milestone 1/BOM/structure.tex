%----------------------------------------------------------------------------------------
%	REQUIRED PACKAGES AND MISC CONFIGURATIONS
%----------------------------------------------------------------------------------------

\usepackage{graphicx} % Required for including images

\usepackage[
	a4paper, % Change to letterpaper for US Letter
	top=2.5cm, 
	bottom=2.5cm, 
	left=2.5cm, 
	right=2.5cm
]{geometry} % Document margins

\usepackage{fp} % Required for invoice calculations

\usepackage[group-separator={,},group-minimum-digits=4, detect-all]{siunitx} % Required for automatically adding commas to large numbers, delimitting for 4 digits (e.g. 1,000) and using the current font

\usepackage{advdate} % Required for date calculation

\setlength\parindent{0pt} % Stop paragraph indentation

%----------------------------------------------------------------------------------------
%	FONTS
%----------------------------------------------------------------------------------------

\usepackage[utf8]{inputenc} % Required for inputting international characters
\usepackage[T1]{fontenc} % Output font encoding for international characters

\usepackage{tgadventor} % Use the TeX Gyre Adventor font
\renewcommand*\familydefault{\sfdefault} % Set the base font of the document to sans serif

%----------------------------------------------------------------------------------------
%	COLOURS
%----------------------------------------------------------------------------------------

\usepackage{xcolor} % Required for defining and using custom colours

\definecolor{highlightcolour}{HTML}{DF0174} % Colour used for making text stand out
\definecolor{rulecolour}{HTML}{B2BEB5} % Colour used for rules

%----------------------------------------------------------------------------------------
%	TABLES
%----------------------------------------------------------------------------------------

\usepackage{colortbl} % Required for colouring table cells (used for rules)

\usepackage{booktabs} % Required for nicer table rules

\usepackage{multirow} % Required for allowing cells to take up multiple rows in tables

\usepackage{array} % Required for customizing table spacing and features
\def\arraystretch{1.2} % Table row spacing, 1 is the default

\newcolumntype{R}[1]{>{\raggedleft\let\newline\\\arraybackslash\hspace{0pt}}m{#1}} % Define a fixed-width right-aligned column type

%----------------------------------------------------------------------------------------
%	CUSTOM COMMANDS
%----------------------------------------------------------------------------------------

\newcommand{\payeename}[1]{\renewcommand{\payeename}{#1}}
\newcommand{\payeejob}[1]{\renewcommand{\payeejob}{#1}}
\newcommand{\payeeaddresslineone}[1]{\renewcommand{\payeeaddresslineone}{#1}}
\newcommand{\payeeaddresslinetwo}[1]{\renewcommand{\payeeaddresslinetwo}{#1}}
\newcommand{\payeecontactlineone}[1]{\renewcommand{\payeecontactlineone}{#1}}
\newcommand{\payeecontactlinetwo}[1]{\renewcommand{\payeecontactlinetwo}{#1}}

\newcommand{\invoiceref}[1]{\renewcommand{\invoiceref}{#1}}
\newcommand{\invoiceissued}[1]{\renewcommand{\invoiceissued}{#1}}
\newcommand{\invoicedue}[1]{\renewcommand{\invoicedue}{#1}}
\newcommand{\projectname}[1]{\renewcommand{\projectname}{#1}}

\newcommand{\companyname}[1]{\renewcommand{\companyname}{#1}}
\newcommand{\sendername}[1]{\renewcommand{\sendername}{#1}}
\newcommand{\senderjob}[1]{\renewcommand{\senderjob}{#1}}
\newcommand{\senderaddresslineone}[1]{\renewcommand{\senderaddresslineone}{#1}}
\newcommand{\senderaddresslinetwo}[1]{\renewcommand{\senderaddresslinetwo}{#1}}
\newcommand{\sendercontactlineone}[1]{\renewcommand{\sendercontactlineone}{#1}}
\newcommand{\sendercontactlinetwo}[1]{\renewcommand{\sendercontactlinetwo}{#1}}

\newcommand{\termsandconditions}[1]{\renewcommand{\termsandconditions}{#1}}

%------------------------------------------------

% Running variables
\gdef\variableA{0}
\gdef\variableB{0}
\gdef\variableC{0}
\gdef\variableD{0}

% Cumulative variables
\gdef\TotalA{0} % Total before tax
\gdef\TotalB{0} % Tax
\gdef\TotalC{0} % Net after tax

\newcommand{\taxrate}[1]{\renewcommand{\taxrate}{#1}} % Tax rate used to automatically calculate tax

%----------------------------------------------------------------------------------------
%	INVOICE TABLES
%----------------------------------------------------------------------------------------

% Payee information (top table)
\newcommand{\invoicedtotable}{
	{
		\footnotesize
		\begin{tabular}{p{0.25\textwidth} p{0.21\textwidth} p{0.21\textwidth} p{0.22\textwidth}}
			\arrayrulecolor{rulecolour}\toprule[0.5pt] % Horizontal line at the top of the table
			\multirow{3}{*}{{\color{highlightcolour} \Huge INVOICE}} & \textbf{Recipient} & \\
			& \payeename & \payeeaddresslineone & \payeecontactlineone \\ 
			& \payeejob & \payeeaddresslinetwo & \payeecontactlinetwo \\
			\arrayrulecolor{rulecolour}\bottomrule[0.5pt] % Horizontal line at the bottom of the table
		\end{tabular}
	}
}

%------------------------------------------------

% Invoice information table
\newcommand{\invoiceinformation}{
	{
		\footnotesize
		\begin{tabular}{p{0.225\textwidth} p{0.225\textwidth} p{0.225\textwidth} p{0.225\textwidth}}
			\textbf{Invoice Number} & \textbf{Date} & \textbf{Payment Due} & \textbf{Project Name} \\
			\arrayrulecolor{rulecolour}\toprule[0.5pt] % Horizontal line
			\invoiceref & \invoiceissued & \invoicedue & \projectname \\
		\end{tabular}
	}
}

%------------------------------------------------

% Invoice items table
\newenvironment{invoicetable}
{
	\begin{tabular}{p{0.45\textwidth} R{0.15\textwidth} R{0.15\textwidth} R{0.15\textwidth}}
		\textbf{ITEM} & \textbf{QUANTITY} & \textbf{RATE} & \textbf{SUBTOTAL} \\
		\arrayrulecolor{rulecolour}\toprule[0.5pt]
}{	
		\arrayrulecolor{rulecolour}\bottomrule[0.5pt]
		\\ [-1em] % Reduce whitespace before the totals
		&& Gross & \$\num{\TotalA} \\
		&& Tax & \$\num{\TotalB} \\ % To display the tax rate in brackets, add the following after "Tax" on this line: (\FPeval{\taxpercent}{round(\taxrate * 100, 2)}\taxpercent\%)
		&& Net & \$\num{\TotalC} \\
	\end{tabular}
}

%------------------------------------------------

% Terms and conditions and total amount due table
\newcommand{\amountdue}{
	{
		\footnotesize
		\begin{tabular}{>{\arraybackslash}m{0.625\textwidth} p{0.05\textwidth} R{0.25\textwidth}}
			\textbf{Terms and Conditions} & & \textbf{Amount Due} \\
			\arrayrulecolor{rulecolour}\toprule[0.5pt]\\[-1.25em] % Horizontal line at the top of the table
			\termsandconditions & & {\color{highlightcolour} \Huge \$\num{\TotalC}}\\
		\end{tabular}
	}
}

%------------------------------------------------

% Sender contact details table
\newcommand{\contactdetails}{
	{
		\footnotesize
		\begin{tabular}{p{0.22\textwidth} p{0.22\textwidth} p{0.23\textwidth} p{0.22\textwidth}}
			\arrayrulecolor{rulecolour}\toprule[0.5pt] % Horizontal line at the top of the table
			\textbf{Sender} & & & \multirow{3}{0.22\textwidth}{\hfill{\color{highlightcolour}\Huge THANKS}}\\
			\sendername & \senderaddresslineone & \sendercontactlineone & \\ 
			\senderjob & \senderaddresslinetwo & \sendercontactlinetwo & \\
			\arrayrulecolor{rulecolour}\bottomrule[0.5pt] % Horizontal line at the bottom of the table
		\end{tabular}
	}
}

%----------------------------------------------------------------------------------------
%	INVOICE ITEM LINES
%----------------------------------------------------------------------------------------

\newcommand{\invoiceitem}[3]{

	% Calculation of running variables (current line)
	\FPmul\gross{#2}{#3}\FPround\gross{\gross}{2} % Calculate the current gross (quantity * rate)
	\global\let\variableA\gross % Set variable for use in table and other calculations
	
	%------------------------------------------------
	
	% Calculation of cumulative variables (total for invoice)
	\FPeval{\beforetax}{round(\TotalA + \variableA, 2)} % Add the current before tax value to the previous total
	\global\let\TotalA\beforetax % Set variable for display at the end and in other calculations
	
	\FPeval{\tax}{round(\TotalA * \taxrate, 2)} % Calculate the updated total tax to pay
	\global\let\TotalB\tax % Set variable for display at the end and in other calculations
	
	\FPeval{\aftertax}{round(\TotalA + \TotalB, 2)} % Calculate the updated total amount due
	\global\let\TotalC\aftertax % Set variable for display at the end
	
	%------------------------------------------------
	
	% Output the table row
	#1 & % Item name
	\num{#2} & % Quantity
	\$\num{#3} & % Rate
	\$\num{\variableA} \\  % Subtotal
}

%----------------------------------------------------------------------------------------
%	INVOICE ENVIRONMENT
%----------------------------------------------------------------------------------------

\newenvironment{invoice}
{
    	\thispagestyle{empty} % Suppress headers and footers

	\begin{center}
		{\color{highlightcolour}\Huge\companyname}
	\end{center}
	
	\vspace{1cm} % Whitespace before the payee information
	
	\invoicedtotable % Payee table (whom the invoice is addressed to)
	\vfill
	\invoiceinformation % Invoice information table
	\vfill
}{ 
	\vfill
	\amountdue % Terms and conditions and total amount due table
	\vfill
	\contactdetails % Contact details table
}
