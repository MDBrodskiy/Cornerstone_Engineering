%%%%%%%%%%%%%%%%%%%%%%%%%%%%%%%%%%%%%%%%%%%%%%%%%%%%%%%%%%%%%%%%%%%%%%%%%%%%%%%%%%%%%%%%%%%%%%%%%%%%%%%%%%%%%%%%%%%%%%%%%%%%%%%%%%%%%%%%%%%%%%%%%%%%%%%%%%%%%%%%%%%
% Written By Michael Brodskiy
% Class: Cornerstone Engineering 1 & 2 (GE1501 & GE1502)
% Professor: B. O'Connell
%%%%%%%%%%%%%%%%%%%%%%%%%%%%%%%%%%%%%%%%%%%%%%%%%%%%%%%%%%%%%%%%%%%%%%%%%%%%%%%%%%%%%%%%%%%%%%%%%%%%%%%%%%%%%%%%%%%%%%%%%%%%%%%%%%%%%%%%%%%%%%%%%%%%%%%%%%%%%%%%%%%

%%%%%%%%%%%%%%%%%%%%%%%%%%%%%%%%%%%%%%%%%%%%%%%%%%%%%%%%%%%%%%%%%%%%%%%%%%%%%%%%%%%%%%%%%%%%%%%%%%%%%%%%%%%%%%%%%%%%%%%%%%%%%%%%%%%%%%%%%%%%%%%%%%%%%%%%%%%%%%%%%%%
% Written By Michael Brodskiy
% Class: Cornerstone Engineering 1 & 2 (GE1501 & GE1502)
% Professor: B. O'Connell
%%%%%%%%%%%%%%%%%%%%%%%%%%%%%%%%%%%%%%%%%%%%%%%%%%%%%%%%%%%%%%%%%%%%%%%%%%%%%%%%%%%%%%%%%%%%%%%%%%%%%%%%%%%%%%%%%%%%%%%%%%%%%%%%%%%%%%%%%%%%%%%%%%%%%%%%%%%%%%%%%%%

\documentclass[12pt]{article} 
\usepackage{alphalph}
\usepackage[utf8]{inputenc}
\usepackage[russian,english]{babel}
\usepackage{titling}
\usepackage{amsmath}
\usepackage{graphicx}
\usepackage{enumitem}
\usepackage{amssymb}
\usepackage[super]{nth}
\usepackage{everysel}
\usepackage{ragged2e}
\usepackage{geometry}
\usepackage{multicol}
\usepackage{fancyhdr}
\usepackage{cancel}
\usepackage{siunitx}
\geometry{top=1.0in,bottom=1.0in,left=1.0in,right=1.0in}
\newcommand{\subtitle}[1]{%
  \posttitle{%
    \par\end{center}
    \begin{center}\large#1\end{center}
    \vskip0.5em}%

}
\usepackage{hyperref}
\hypersetup{
colorlinks=true,
linkcolor=blue,
filecolor=magenta,      
urlcolor=blue,
citecolor=blue,
}


\title{The Concentric Tube Launcher}
\date{\today}
\author{Michael Brodskiy\\ \small Professor: B. O'Connell}

\begin{document}

\maketitle

\begin{figure}[h!]
  \centering
  \tikzset{every picture/.style={line width=0.75pt}} %set default line width to 0.75pt        

\begin{tikzpicture}[x=0.75pt,y=0.75pt,yscale=-1,xscale=1]
%uncomment if require: \path (0,471); %set diagram left start at 0, and has height of 471

%Shape: Diagonal Stripe [id:dp44643171889307887] 
\draw   (227.02,344.17) -- (227.02,391) -- (157.02,220.19) -- (176.21,220.19) -- cycle ;
%Shape: Arc [id:dp3829721896258529] 
\draw  [draw opacity=0][dash pattern={on 4.5pt off 4.5pt}] (64.5,147.79) .. controls (64.5,147.79) and (64.5,147.79) .. (64.5,147.79) .. controls (84.4,147.79) and (100.53,164.3) .. (100.53,184.67) .. controls (100.53,205.04) and (84.4,221.56) .. (64.5,221.56) -- (64.5,184.67) -- cycle ; \draw  [dash pattern={on 4.5pt off 4.5pt}] (64.5,147.79) .. controls (64.5,147.79) and (64.5,147.79) .. (64.5,147.79) .. controls (84.4,147.79) and (100.53,164.3) .. (100.53,184.67) .. controls (100.53,205.04) and (84.4,221.56) .. (64.5,221.56) ;  
%Straight Lines [id:da6447442957821032] 
\draw  [dash pattern={on 4.5pt off 4.5pt}]  (372.78,157.85) -- (217.1,157.36) ;
%Shape: Boxed Line [id:dp9137077542810423] 
\draw    (438.67,157.36) -- (372.78,157.85) ;
%Shape: Arc [id:dp31065750279651816] 
\draw  [draw opacity=0] (439.47,207.64) .. controls (426.5,207.63) and (415.99,196.37) .. (415.99,182.49) .. controls (415.99,168.92) and (426.04,157.86) .. (438.61,157.36) -- (439.49,182.49) -- cycle ; \draw   (439.47,207.64) .. controls (426.5,207.63) and (415.99,196.37) .. (415.99,182.49) .. controls (415.99,168.92) and (426.04,157.86) .. (438.61,157.36) ;  
%Shape: Arc [id:dp7246623187905588] 
\draw  [draw opacity=0] (437.77,157.41) .. controls (438.34,157.37) and (438.91,157.34) .. (439.49,157.34) .. controls (452.47,157.34) and (462.99,168.6) .. (462.99,182.49) .. controls (462.99,196.38) and (452.47,207.64) .. (439.49,207.64) .. controls (435.66,207.64) and (432.04,206.66) .. (428.84,204.91) -- (439.49,182.49) -- cycle ; \draw   (437.77,157.41) .. controls (438.34,157.37) and (438.91,157.34) .. (439.49,157.34) .. controls (452.47,157.34) and (462.99,168.6) .. (462.99,182.49) .. controls (462.99,196.38) and (452.47,207.64) .. (439.49,207.64) .. controls (435.66,207.64) and (432.04,206.66) .. (428.84,204.91) ;  
%Shape: Arc [id:dp8321391507112232] 
\draw  [draw opacity=0] (217.91,207.64) .. controls (204.94,207.63) and (194.43,196.37) .. (194.43,182.49) .. controls (194.43,168.92) and (204.48,157.86) .. (217.05,157.36) -- (217.93,182.49) -- cycle ; \draw   (217.91,207.64) .. controls (204.94,207.63) and (194.43,196.37) .. (194.43,182.49) .. controls (194.43,168.92) and (204.48,157.86) .. (217.05,157.36) ;  
%Shape: Arc [id:dp5772023197639267] 
\draw  [draw opacity=0] (424.79,207.52) .. controls (418.22,214.62) and (408.93,219.04) .. (398.63,219.04) .. controls (378.73,219.04) and (362.6,202.53) .. (362.6,182.16) .. controls (362.6,161.79) and (378.73,145.27) .. (398.63,145.27) .. controls (409.3,145.27) and (418.88,150.02) .. (425.48,157.55) -- (398.63,182.16) -- cycle ; \draw   (424.79,207.52) .. controls (418.22,214.62) and (408.93,219.04) .. (398.63,219.04) .. controls (378.73,219.04) and (362.6,202.53) .. (362.6,182.16) .. controls (362.6,161.79) and (378.73,145.27) .. (398.63,145.27) .. controls (409.3,145.27) and (418.88,150.02) .. (425.48,157.55) ;  
%Straight Lines [id:da8276168802295119] 
\draw    (439.47,207.64) -- (373.59,208.13) ;
%Straight Lines [id:da8086723755554028] 
\draw  [dash pattern={on 4.5pt off 4.5pt}]  (373.59,208.13) -- (217.91,207.64) ;
%Shape: Boxed Line [id:dp04417146350403023] 
\draw    (396.84,145.27) -- (286.06,146.11) ;
%Shape: Boxed Line [id:dp4650146603828804] 
\draw    (286.06,146.11) -- (175.28,146.95) ;
%Shape: Boxed Line [id:dp2321133945044378] 
\draw    (389.79,218.2) -- (279.01,219.04) ;
%Shape: Boxed Line [id:dp1511670848453761] 
\draw    (279.01,219.04) -- (168.23,219.88) ;
%Shape: Moon [id:dp8540543539810237] 
\draw  [fill={rgb, 255:red, 255; green, 255; blue, 255 }  ,fill opacity=1 ][dash pattern={on 4.5pt off 4.5pt}] (227.01,218.36) .. controls (205.81,218.36) and (188.63,202.41) .. (188.63,182.74) .. controls (188.63,163.06) and (205.81,147.11) .. (227.01,147.11) .. controls (210.44,151.21) and (198.22,165.61) .. (198.22,182.74) .. controls (198.22,199.87) and (210.44,214.27) .. (227.01,218.36) -- cycle ;
%Shape: Moon [id:dp5882029644966247] 
\draw  [fill={rgb, 255:red, 255; green, 255; blue, 255 }  ,fill opacity=1 ][dash pattern={on 4.5pt off 4.5pt}] (159.84,147.11) .. controls (181.04,147.11) and (198.22,163.06) .. (198.22,182.74) .. controls (198.22,202.41) and (181.04,218.36) .. (159.84,218.36) .. controls (176.41,214.27) and (188.63,199.87) .. (188.63,182.74) .. controls (188.63,165.61) and (176.41,151.21) .. (159.84,147.11) -- cycle ;
%Shape: Boxed Line [id:dp6579420712092399] 
\draw    (175.28,146.95) -- (64.5,147.79) ;
%Shape: Boxed Line [id:dp19771990599738065] 
\draw    (168.23,219.88) -- (64.5,221.56) ;
%Shape: Boxed Line [id:dp09246533962136039] 
\draw    (512.98,207.31) -- (441.47,207.63) ;
\draw [shift={(439.47,207.64)}, rotate = 359.74] [color={rgb, 255:red, 0; green, 0; blue, 0 }  ][line width=0.75]    (10.93,-3.29) .. controls (6.95,-1.4) and (3.31,-0.3) .. (0,0) .. controls (3.31,0.3) and (6.95,1.4) .. (10.93,3.29)   ;
%Curve Lines [id:da06123511321721531] 
\draw [line width=2.25]    (457,134) .. controls (496.77,148.55) and (481.03,163.1) .. (443.07,180.84) ;
\draw [shift={(439.49,182.49)}, rotate = 335.47] [color={rgb, 255:red, 0; green, 0; blue, 0 }  ][line width=2.25]    (17.49,-5.26) .. controls (11.12,-2.23) and (5.29,-0.48) .. (0,0) .. controls (5.29,0.48) and (11.12,2.23) .. (17.49,5.26)   ;
%Shape: Boxed Line [id:dp04823339202175414] 
\draw    (298.22,282.74) -- (199.64,184.15) ;
\draw [shift={(198.22,182.74)}, rotate = 45] [color={rgb, 255:red, 0; green, 0; blue, 0 }  ][line width=0.75]    (10.93,-3.29) .. controls (6.95,-1.4) and (3.31,-0.3) .. (0,0) .. controls (3.31,0.3) and (6.95,1.4) .. (10.93,3.29)   ;
%Shape: Moon [id:dp24174634531764227] 
\draw  [fill={rgb, 255:red, 255; green, 255; blue, 255 }  ,fill opacity=1 ][dash pattern={on 4.5pt off 4.5pt}] (221.01,218.36) .. controls (137.06,218.36) and (69,202.41) .. (69,182.74) .. controls (69,163.06) and (137.06,147.11) .. (221.01,147.11) .. controls (147.9,149.75) and (92,164.7) .. (92,182.74) .. controls (92,200.78) and (147.9,215.72) .. (221.01,218.36) -- cycle ;
%Shape: Boxed Line [id:dp9421913663732036] 
\draw    (188.63,182.74) -- (94,183.72) ;
\draw [shift={(92,183.74)}, rotate = 359.41] [color={rgb, 255:red, 0; green, 0; blue, 0 }  ][line width=0.75]    (10.93,-3.29) .. controls (6.95,-1.4) and (3.31,-0.3) .. (0,0) .. controls (3.31,0.3) and (6.95,1.4) .. (10.93,3.29)   ;
%Shape: Block Arc [id:dp2671154342059705] 
\draw   (225.99,254.35) .. controls (207.41,253.82) and (192.45,238.8) .. (192.02,220.19) -- (200,220) .. controls (200.33,234.37) and (211.88,245.96) .. (226.22,246.37) -- cycle ;
%Shape: Right Triangle [id:dp12222925013676056] 
\draw  [fill={rgb, 255:red, 0; green, 0; blue, 0 }  ,fill opacity=1 ] (200,180) -- (192.02,220.19) -- (200,220.19) -- cycle ;
%Straight Lines [id:da6026957076507713] 
\draw [line width=2.25]    (75.8,294.45) -- (188.65,222.34) ;
\draw [shift={(192.02,220.19)}, rotate = 147.42] [color={rgb, 255:red, 0; green, 0; blue, 0 }  ][line width=2.25]    (17.49,-5.26) .. controls (11.12,-2.23) and (5.29,-0.48) .. (0,0) .. controls (5.29,0.48) and (11.12,2.23) .. (17.49,5.26)   ;
%Straight Lines [id:da09418100379889882] 
\draw    (396.84,145.27) -- (397,157) ;
%Straight Lines [id:da8937010383750714] 
\draw    (396.63,208.48) -- (396.79,220.2) ;
%Shape: Arc [id:dp6616955505275475] 
\draw  [draw opacity=0] (64.5,221.56) .. controls (64.5,221.56) and (64.5,221.56) .. (64.5,221.56) .. controls (44.6,221.56) and (28.46,205.04) .. (28.46,184.67) .. controls (28.46,164.3) and (44.6,147.79) .. (64.5,147.79) -- (64.5,184.67) -- cycle ; \draw   (64.5,221.56) .. controls (64.5,221.56) and (64.5,221.56) .. (64.5,221.56) .. controls (44.6,221.56) and (28.46,205.04) .. (28.46,184.67) .. controls (28.46,164.3) and (44.6,147.79) .. (64.5,147.79) ;  
%Shape: Block Arc [id:dp10633354848632748] 
\draw   (256,218.92) .. controls (256.01,219.44) and (256.01,219.97) .. (256.01,220.5) .. controls (256.01,255.14) and (234.9,284.47) .. (205.77,294.4) -- (203,284.99) .. controls (228.07,276.25) and (246.21,250.68) .. (246.21,220.5) .. controls (246.21,220.04) and (246.21,219.59) .. (246.2,219.13) -- cycle ;
%Straight Lines [id:da8113082771754814] 
\draw [line width=2.25]    (188,81) -- (72.04,180.14) ;
\draw [shift={(69,182.74)}, rotate = 319.47] [color={rgb, 255:red, 0; green, 0; blue, 0 }  ][line width=2.25]    (17.49,-5.26) .. controls (11.12,-2.23) and (5.29,-0.48) .. (0,0) .. controls (5.29,0.48) and (11.12,2.23) .. (17.49,5.26)   ;

% Text Node
\draw (514.4,209.95) node [anchor=north] [inner sep=0.75pt]  [rotate=-331.77] [align=left] {1. Inner barrel collapses\\for single-handed reloading};
% Text Node
\draw (455,131) node [anchor=south east] [inner sep=0.75pt]   [align=left] {4. Projectile is loaded\\into inner barrel};
% Text Node
\draw (293.56,285.09) node [anchor=north west][inner sep=0.75pt]  [rotate=-11.92] [align=left] {2. Collapsing inner barrel\\stretches rubber bands};
% Text Node
\draw (76.06,298.05) node [anchor=north west][inner sep=0.75pt]  [rotate=-29.52] [align=left] {3. Hook, attached to \\trigger, is pulled with bands};
% Text Node
\draw (188,78) node [anchor=south] [inner sep=0.75pt]   [align=left] {5. Pulling trigger releases hook,\\firing the loaded projectile};


\end{tikzpicture}

  \caption{Concentric Tube Launcher Concept Sketch}
  \label{fig:1}
\end{figure}

\paragraph{} The concentric tube launcher contains an inner and outer barrel for single-handed reloading. Pressing down on the inner barrel collapses it, pushing back two rubber bands. These bands are held in place by a hook attached to a trigger. A ping pong ball is then loaded into the inner barrel. Pulling the trigger releases the bands, propelling the ball, which is loose inside the inner barrel, and launching it.

\newline
\vspace{25pt}

Source:

\text{[1]} Vanek and Flatau, “Launcher for use in toy gun, has trigger that is moved for causing drive to rotate drive unit for contacting ring airfoil projectile, where drive unit is configured to engage projectile for launching projectile,” Jan. 07, 2005 Accessed: Sep. 24, 2022. [Online]. Available: \href{https://www-webofscience-com.ezproxy.neu.edu/wos/alldb/full-record/DIIDW:2005616697}{https://www-webofscience-com.ezproxy.neu.edu/wos/alldb/full-record/DIIDW:2005616697}\\
\text{[2]} R. M. French, V. Gorrepati, E. Alcorta, and M. Jackson, “THE MECHANICS OF A PING-PONG BALL GUN,” Experimental Techniques, vol. 32, no. 1, pp. 24–30, Jan. 2008, doi: 10.1111/j.1747-1567.2007.00214.x.


\end{document}

