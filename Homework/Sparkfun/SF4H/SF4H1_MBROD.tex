%%%%%%%%%%%%%%%%%%%%%%%%%%%%%%%%%%%%%%%%%%%%%%%%%%%%%%%%%%%%%%%%%%%%%%%%%%%%%%%%%%%%%%%%%%%%%%%%%%%%%%%%%%%%%%%%%%%%%%%%%%%%%%%%%%%%%%%%%%%%%%%%%%%%%%%%%%%%%%%%%%%
% Written By Michael Brodskiy
% Class: Cornerstone Engineering 1 & 2 (GE1501 & GE1502)
% Professor: B. O'Connell
%%%%%%%%%%%%%%%%%%%%%%%%%%%%%%%%%%%%%%%%%%%%%%%%%%%%%%%%%%%%%%%%%%%%%%%%%%%%%%%%%%%%%%%%%%%%%%%%%%%%%%%%%%%%%%%%%%%%%%%%%%%%%%%%%%%%%%%%%%%%%%%%%%%%%%%%%%%%%%%%%%%

%%%%%%%%%%%%%%%%%%%%%%%%%%%%%%%%%%%%%%%%%%%%%%%%%%%%%%%%%%%%%%%%%%%%%%%%%%%%%%%%%%%%%%%%%%%%%%%%%%%%%%%%%%%%%%%%%%%%%%%%%%%%%%%%%%%%%%%%%%%%%%%%%%%%%%%%%%%%%%%%%%%
% Written By Michael Brodskiy
% Class: Cornerstone Engineering 1 & 2 (GE1501 & GE1502)
% Professor: B. O'Connell
%%%%%%%%%%%%%%%%%%%%%%%%%%%%%%%%%%%%%%%%%%%%%%%%%%%%%%%%%%%%%%%%%%%%%%%%%%%%%%%%%%%%%%%%%%%%%%%%%%%%%%%%%%%%%%%%%%%%%%%%%%%%%%%%%%%%%%%%%%%%%%%%%%%%%%%%%%%%%%%%%%%

\documentclass[12pt]{article} 
\usepackage{alphalph}
\usepackage[utf8]{inputenc}
\usepackage[russian,english]{babel}
\usepackage{titling}
\usepackage{amsmath}
\usepackage{graphicx}
\usepackage{enumitem}
\usepackage{amssymb}
\usepackage[super]{nth}
\usepackage{everysel}
\usepackage{ragged2e}
\usepackage{geometry}
\usepackage{multicol}
\usepackage{fancyhdr}
\usepackage{cancel}
\usepackage{siunitx}
\geometry{top=1.0in,bottom=1.0in,left=1.0in,right=1.0in}
\newcommand{\subtitle}[1]{%
  \posttitle{%
    \par\end{center}
    \begin{center}\large#1\end{center}
    \vskip0.5em}%

}
\usepackage{hyperref}
\hypersetup{
colorlinks=true,
linkcolor=blue,
filecolor=magenta,      
urlcolor=blue,
citecolor=blue,
}


\title{SF4H1 Pseudocode}
\date{\today}
\author{Michael Brodskiy\\ \small Professor: B. O'Connell}

\begin{document}

\maketitle

    \begin{algorithm}
      \caption{SF4H1}\label{SF4H1}
      \begin{algorithmic}[1]
        \Procedure{Temperature Sensor with Thermostat Control}{}
        \State Define necessary variables and port numbers
        \State Set screen contrast
        \State Set cursor to first line
        \State Print first line
        \State Set cursor to second line
        \State Print second line
        \State Delay 2500 ms
        \State Clear screen
        \State Set cursor to first line
        \State Print first line
        \State Set cursor to second line
        \State Print second line
        \State Delay 2500 ms
        \State Begin loop
        \If {Button has been pressed AND Not the same as previous button state AND More than 100 ms has passed}
            \State Switch units
            \State Print units
        \EndIf
        \If {In Celsius mode AND More than 5000 ms have passed}
            \State Print Celsius
            \State Restart timer
        \ElsIf {In Fahrenheit mode AND More than 5000 ms have passed}
            \State Print Fahrenheit
            \State Restart timer
        \EndIf
        \State Print thermostat temperature on second line
        \If {Thermostat temperature is greater than threshold temperature}
            \State Buzz one tone for 500 ms
            \State Buzz other tone for 500 ms
            \State Restart timer
        \EndIf
        \EndProcedure
      \end{algorithmic}
    \end{algorithm}

\end{document}

