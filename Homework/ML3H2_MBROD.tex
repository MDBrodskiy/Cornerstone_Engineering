%%%%%%%%%%%%%%%%%%%%%%%%%%%%%%%%%%%%%%%%%%%%%%%%%%%%%%%%%%%%%%%%%%%%%%%%%%%%%%%%%%%%%%%%%%%%%%%%%%%%%%%%%%%%%%%%%%%%%%%%%%%%%%%%%%%%%%%%%%%%%%%%%%%%%%%%%%%%%%%%%%%
% Written By Michael Brodskiy
% Class: Cornerstone Engineering 1 & 2 (GE1501 & GE1502)
% Professor: B. O'Connell
%%%%%%%%%%%%%%%%%%%%%%%%%%%%%%%%%%%%%%%%%%%%%%%%%%%%%%%%%%%%%%%%%%%%%%%%%%%%%%%%%%%%%%%%%%%%%%%%%%%%%%%%%%%%%%%%%%%%%%%%%%%%%%%%%%%%%%%%%%%%%%%%%%%%%%%%%%%%%%%%%%%

%%%%%%%%%%%%%%%%%%%%%%%%%%%%%%%%%%%%%%%%%%%%%%%%%%%%%%%%%%%%%%%%%%%%%%%%%%%%%%%%%%%%%%%%%%%%%%%%%%%%%%%%%%%%%%%%%%%%%%%%%%%%%%%%%%%%%%%%%%%%%%%%%%%%%%%%%%%%%%%%%%%
% Written By Michael Brodskiy
% Class: Cornerstone Engineering 1 & 2 (GE1501 & GE1502)
% Professor: B. O'Connell
%%%%%%%%%%%%%%%%%%%%%%%%%%%%%%%%%%%%%%%%%%%%%%%%%%%%%%%%%%%%%%%%%%%%%%%%%%%%%%%%%%%%%%%%%%%%%%%%%%%%%%%%%%%%%%%%%%%%%%%%%%%%%%%%%%%%%%%%%%%%%%%%%%%%%%%%%%%%%%%%%%%

\documentclass[12pt]{article} 
\usepackage{alphalph}
\usepackage[utf8]{inputenc}
\usepackage[russian,english]{babel}
\usepackage{titling}
\usepackage{amsmath}
\usepackage{graphicx}
\usepackage{enumitem}
\usepackage{amssymb}
\usepackage[super]{nth}
\usepackage{everysel}
\usepackage{ragged2e}
\usepackage{geometry}
\usepackage{multicol}
\usepackage{fancyhdr}
\usepackage{cancel}
\usepackage{siunitx}
\geometry{top=1.0in,bottom=1.0in,left=1.0in,right=1.0in}
\newcommand{\subtitle}[1]{%
  \posttitle{%
    \par\end{center}
    \begin{center}\large#1\end{center}
    \vskip0.5em}%

}
\usepackage{hyperref}
\hypersetup{
colorlinks=true,
linkcolor=blue,
filecolor=magenta,      
urlcolor=blue,
citecolor=blue,
}


\title{ML3H2 Flowchart \& Explanation}
\date{\today}
\author{Michael Brodskiy\\ \small Professor: B. O'Connell}

\begin{document}

\maketitle

\begin{justify}

  The idea behind this demo is that it compares several of the variables involved to reaction time, while still giving the user some control over the data. There are two inputs, \textsc{hand}, which sets the handedness of the user as a constraint, and \textsc{N}, which limits the data arrays to size \textsc{N}\footnote{NOTE: setting \textsc{N} too large causes a mismatch in array sizes}. Four tiled scatterplots are generated: Return time, first move time, moving time, and age versus reaction time. The attached example runs \textsc{ML3H2\_MBROD(``r'', 50)}. See below for a flowchart.
\end{justify}

\begin{figure}[H]
  \centering
  \tikzset{every picture/.style={line width=0.75pt}} %set default line width to 0.75pt        

\begin{tikzpicture}[x=0.75pt,y=0.75pt,yscale=-1,xscale=1]
%uncomment if require: \path (0,801); %set diagram left start at 0, and has height of 801

%Shape: Ellipse [id:dp06026735517165571] 
\draw   (229,39.5) .. controls (229,18.79) and (271.09,2) .. (323,2) .. controls (374.91,2) and (417,18.79) .. (417,39.5) .. controls (417,60.21) and (374.91,77) .. (323,77) .. controls (271.09,77) and (229,60.21) .. (229,39.5) -- cycle ;
%Shape: Rectangle [id:dp880516595801381] 
\draw   (248,151) -- (398,151) -- (398,226) -- (248,226) -- cycle ;
%Straight Lines [id:da08375757950130125] 
\draw    (323,77) -- (323,148) ;
\draw [shift={(323,150)}, rotate = 270] [color={rgb, 255:red, 0; green, 0; blue, 0 }  ][line width=0.75]    (10.93,-3.29) .. controls (6.95,-1.4) and (3.31,-0.3) .. (0,0) .. controls (3.31,0.3) and (6.95,1.4) .. (10.93,3.29)   ;
%Flowchart: Decision [id:dp5735470816878454] 
\draw   (323,446) -- (398.5,484.5) -- (323,523) -- (247.5,484.5) -- cycle ;
%Straight Lines [id:da5891435380987948] 
\draw    (323,227) -- (323,298) ;
\draw [shift={(323,300)}, rotate = 270] [color={rgb, 255:red, 0; green, 0; blue, 0 }  ][line width=0.75]    (10.93,-3.29) .. controls (6.95,-1.4) and (3.31,-0.3) .. (0,0) .. controls (3.31,0.3) and (6.95,1.4) .. (10.93,3.29)   ;
%Shape: Rectangle [id:dp1739230692817313] 
\draw   (248,299) -- (398,299) -- (398,374) -- (248,374) -- cycle ;
%Straight Lines [id:da5912825997995481] 
\draw    (323,373) -- (323,444) ;
\draw [shift={(323,446)}, rotate = 270] [color={rgb, 255:red, 0; green, 0; blue, 0 }  ][line width=0.75]    (10.93,-3.29) .. controls (6.95,-1.4) and (3.31,-0.3) .. (0,0) .. controls (3.31,0.3) and (6.95,1.4) .. (10.93,3.29)   ;
%Shape: Boxed Line [id:dp9932206661848584] 
\draw    (398.5,484.5) -- (469.5,484.5) ;
\draw [shift={(471.5,484.5)}, rotate = 180] [color={rgb, 255:red, 0; green, 0; blue, 0 }  ][line width=0.75]    (10.93,-3.29) .. controls (6.95,-1.4) and (3.31,-0.3) .. (0,0) .. controls (3.31,0.3) and (6.95,1.4) .. (10.93,3.29)   ;
%Flowchart: Decision [id:dp4743967302886527] 
\draw   (545,446) -- (620.5,484.5) -- (545,523) -- (469.5,484.5) -- cycle ;
%Shape: Boxed Line [id:dp2787171518355003] 
\draw    (545,446) -- (545,375) ;
\draw [shift={(545,373)}, rotate = 90] [color={rgb, 255:red, 0; green, 0; blue, 0 }  ][line width=0.75]    (10.93,-3.29) .. controls (6.95,-1.4) and (3.31,-0.3) .. (0,0) .. controls (3.31,0.3) and (6.95,1.4) .. (10.93,3.29)   ;
%Shape: Rectangle [id:dp9152323990484745] 
\draw   (470,151) -- (620,151) -- (620,226) -- (470,226) -- cycle ;
%Flowchart: Decision [id:dp7823778282166178] 
\draw   (323,566) -- (398.5,604.5) -- (323,643) -- (247.5,604.5) -- cycle ;
%Straight Lines [id:da1962915410506696] 
\draw    (641,605) -- (400.5,604.5) ;
\draw [shift={(398.5,604.5)}, rotate = 0.12] [color={rgb, 255:red, 0; green, 0; blue, 0 }  ][line width=0.75]    (10.93,-3.29) .. controls (6.95,-1.4) and (3.31,-0.3) .. (0,0) .. controls (3.31,0.3) and (6.95,1.4) .. (10.93,3.29)   ;
%Straight Lines [id:da8225648760923123] 
\draw    (323,523) -- (323,564) ;
\draw [shift={(323,566)}, rotate = 270] [color={rgb, 255:red, 0; green, 0; blue, 0 }  ][line width=0.75]    (10.93,-3.29) .. controls (6.95,-1.4) and (3.31,-0.3) .. (0,0) .. controls (3.31,0.3) and (6.95,1.4) .. (10.93,3.29)   ;
%Straight Lines [id:da8422962655605244] 
\draw    (620.5,335.5) -- (642.5,335.5) ;
%Straight Lines [id:da04998585252964416] 
\draw    (642,334) -- (641,605) ;
%Straight Lines [id:da43416796211710285] 
\draw    (143,605) -- (247.5,604.5) ;
%Straight Lines [id:da8807304940200649] 
\draw    (145,410) -- (143,605) ;
%Straight Lines [id:da7168272354961771] 
\draw    (145,410) -- (321,409.51) ;
\draw [shift={(323,409.5)}, rotate = 179.84] [color={rgb, 255:red, 0; green, 0; blue, 0 }  ][line width=0.75]    (10.93,-3.29) .. controls (6.95,-1.4) and (3.31,-0.3) .. (0,0) .. controls (3.31,0.3) and (6.95,1.4) .. (10.93,3.29)   ;
%Straight Lines [id:da086039166652641] 
\draw    (620.5,484.5) -- (642.5,484.5) ;
%Shape: Circle [id:dp8414382717513849] 
\draw  [fill={rgb, 255:red, 0; green, 0; blue, 0 }  ,fill opacity=1 ] (637.5,484.5) .. controls (637.5,482.57) and (639.07,481) .. (641,481) .. controls (642.93,481) and (644.5,482.57) .. (644.5,484.5) .. controls (644.5,486.43) and (642.93,488) .. (641,488) .. controls (639.07,488) and (637.5,486.43) .. (637.5,484.5) -- cycle ;
%Straight Lines [id:da878855301686359] 
\draw    (323,643) -- (323,684) ;
\draw [shift={(323,686)}, rotate = 270] [color={rgb, 255:red, 0; green, 0; blue, 0 }  ][line width=0.75]    (10.93,-3.29) .. controls (6.95,-1.4) and (3.31,-0.3) .. (0,0) .. controls (3.31,0.3) and (6.95,1.4) .. (10.93,3.29)   ;
%Flowchart: Terminator [id:dp2386338233145222] 
\draw   (272.16,685) -- (374.84,685) .. controls (388.18,685) and (399,701.57) .. (399,722) .. controls (399,742.43) and (388.18,759) .. (374.84,759) -- (272.16,759) .. controls (258.82,759) and (248,742.43) .. (248,722) .. controls (248,701.57) and (258.82,685) .. (272.16,685) -- cycle ;
%Flowchart: Decision [id:dp008358458821891013] 
\draw   (545,297) -- (620.5,335.5) -- (545,374) -- (469.5,335.5) -- cycle ;
%Straight Lines [id:da32593178252161925] 
\draw    (643,187.5) -- (642,334) ;
%Straight Lines [id:da6520057997434436] 
\draw    (621,187.5) -- (643,187.5) ;
%Shape: Boxed Line [id:dp33684862453830156] 
\draw    (545,299) -- (545,228) ;
\draw [shift={(545,226)}, rotate = 90] [color={rgb, 255:red, 0; green, 0; blue, 0 }  ][line width=0.75]    (10.93,-3.29) .. controls (6.95,-1.4) and (3.31,-0.3) .. (0,0) .. controls (3.31,0.3) and (6.95,1.4) .. (10.93,3.29)   ;
%Shape: Circle [id:dp9060227765996967] 
\draw  [fill={rgb, 255:red, 0; green, 0; blue, 0 }  ,fill opacity=1 ] (639,335.5) .. controls (639,333.57) and (640.57,332) .. (642.5,332) .. controls (644.43,332) and (646,333.57) .. (646,335.5) .. controls (646,337.43) and (644.43,339) .. (642.5,339) .. controls (640.57,339) and (639,337.43) .. (639,335.5) -- cycle ;

% Text Node
\draw (323,39.5) node  [font=\tiny] [align=left] {{\footnotesize ML3H2\_MBROD(\textit{hand}, \textit{N})}};
% Text Node
\draw (323,188.5) node  [font=\scriptsize] [align=left] {\begin{minipage}[lt]{99.59pt}\setlength\topsep{0pt}
\begin{center}
{\footnotesize Retrieve Course Data, }\\{\footnotesize Organize Arrays, and Create Counters}
\end{center}

\end{minipage}};
% Text Node
\draw (323,604.5) node  [font=\scriptsize] [align=left] {\begin{minipage}[lt]{37.44pt}\setlength\topsep{0pt}
\begin{center}
{\footnotesize More Values?}
\end{center}

\end{minipage}};
% Text Node
\draw (323,336.5) node  [font=\scriptsize] [align=left] {\begin{minipage}[lt]{59.3pt}\setlength\topsep{0pt}
\begin{center}
{\footnotesize Start Looping Through}\\{\footnotesize All Values in the Data}\\{\footnotesize Sets}
\end{center}

\end{minipage}};
% Text Node
\draw (323,484.5) node  [font=\scriptsize] [align=left] {\begin{minipage}[lt]{45.45pt}\setlength\topsep{0pt}
\begin{center}
{\footnotesize Values}\\{\footnotesize less than 7 times}\\{\footnotesize min?}
\end{center}

\end{minipage}};
% Text Node
\draw (545,188.5) node  [font=\scriptsize] [align=left] {\begin{minipage}[lt]{61.92pt}\setlength\topsep{0pt}
\begin{center}
{\footnotesize Add value to respective}\\{\footnotesize array}
\end{center}

\end{minipage}};
% Text Node
\draw (545,488.5) node  [font=\scriptsize] [align=left] {\begin{minipage}[lt]{51.18pt}\setlength\topsep{0pt}
\begin{center}
{\footnotesize Less than \textit{N} values}\\{\footnotesize in array?}
\end{center}

\end{minipage}};
% Text Node
\draw (324,533) node [anchor=north west][inner sep=0.75pt]  [font=\footnotesize] [align=left] {no};
% Text Node
\draw (416,469) node [anchor=north west][inner sep=0.75pt]  [font=\footnotesize] [align=left] {yes};
% Text Node
\draw (523,404) node [anchor=north west][inner sep=0.75pt]  [font=\footnotesize] [align=left] {yes};
% Text Node
\draw (227,590) node [anchor=north west][inner sep=0.75pt]  [font=\footnotesize] [align=left] {yes};
% Text Node
\draw (619,470) node [anchor=north west][inner sep=0.75pt]  [font=\footnotesize] [align=left] {no};
% Text Node
\draw (324,653) node [anchor=north west][inner sep=0.75pt]  [font=\footnotesize] [align=left] {no};
% Text Node
\draw (323.5,722) node  [font=\scriptsize] [align=left] {\begin{minipage}[lt]{48.63pt}\setlength\topsep{0pt}
\begin{center}
{\footnotesize Generate four}\\{\footnotesize plots using filtered}\\{\footnotesize arrays}
\end{center}

\end{minipage}};
% Text Node
\draw (545,335.5) node  [font=\scriptsize] [align=left] {\begin{minipage}[lt]{41.67pt}\setlength\topsep{0pt}
\begin{center}
{\footnotesize Subject hand is}\\{\footnotesize same as \textit{hand}}
\end{center}

\end{minipage}};
% Text Node
\draw (523,257) node [anchor=north west][inner sep=0.75pt]  [font=\footnotesize] [align=left] {yes};
% Text Node
\draw (621,323) node [anchor=north west][inner sep=0.75pt]  [font=\footnotesize] [align=left] {no};


\end{tikzpicture}

  \caption{Plot Flowchart}
  \label{fig:1}
\end{figure}

\end{document}

