\documentclass[conference]{IEEEtran}
\IEEEoverridecommandlockouts
% The preceding line is only needed to identify funding in the first footnote. If that is unneeded, please comment it out.
\usepackage{cite}
\usepackage{amsmath,amssymb,amsfonts}
\usepackage{algorithmic}
\usepackage{graphicx}
\usepackage{textcomp}
\usepackage{xcolor}
\usepackage{hyperref}
\hypersetup{
colorlinks=true,
linkcolor=black,
filecolor=black,      
urlcolor=black,
citecolor=black,
}
\def\BibTeX{{\rm B\kern-.05em{\sc i\kern-.025em b}\kern-.08em
    T\kern-.1667em\lower.7ex\hbox{E}\kern-.125emX}}
\begin{document}

\title{D4 — Engineering Ethics Case Study}

\author{\IEEEauthorblockN{Michael Brodskiy}
\IEEEauthorblockA{\textit{College of Engineering} \\
\textit{Northeastern University}\\
Boston, U.S.A \\
\href{mailto:Brodskiy.M@Northeastern.edu}{Brodskiy.M@Northeastern.edu}}
%\and
%\IEEEauthorblockN{2\textsuperscript{nd} Given Name Surname}
%\IEEEauthorblockA{\textit{dept. name of organization (of Aff.)} \\
%\textit{name of organization (of Aff.)}\\
%City, Country \\
%email address or ORCID}
%\and
%\IEEEauthorblockN{3\textsuperscript{rd} Given Name Surname}
%\IEEEauthorblockA{\textit{dept. name of organization (of Aff.)} \\
%\textit{name of organization (of Aff.)}\\
%City, Country \\
%email address or ORCID}
%\and
%\IEEEauthorblockN{4\textsuperscript{th} Given Name Surname}
%\IEEEauthorblockA{\textit{dept. name of organization (of Aff.)} \\
%\textit{name of organization (of Aff.)}\\
%City, Country \\
%email address or ORCID}
%\and
%\IEEEauthorblockN{5\textsuperscript{th} Given Name Surname}
%\IEEEauthorblockA{\textit{dept. name of organization (of Aff.)} \\
%\textit{name of organization (of Aff.)}\\
%City, Country \\
%email address or ORCID}
%\and
%\IEEEauthorblockN{6\textsuperscript{th} Given Name Surname}
%\IEEEauthorblockA{\textit{dept. name of organization (of Aff.)} \\
%\textit{name of organization (of Aff.)}\\
%City, Country \\
%email address or ORCID}
}

\maketitle

\begin{abstract}
  The purpose of this document is to analyze the engineering ethics behind the Boston Molasses Flood of 1919 and another engineering disaster through the lens of the engineering organization closest to my major (IEEE).
\end{abstract}

\begin{IEEEkeywords}
  \underline{Engineering ethics}, \underline{disaster}, \underline{IEEE}, \underline{Boston Molasses Flood}
\end{IEEEkeywords}

\section{Introduction}
This document is prepared in accordance with the Institute of Electrical and Electronics Engineers (IEEE) specifications, and attempts to address IEEE code of ethics violations in the Boston Molasses Flood and another engineering disaster.

\section{The Boston Molasses Flood}

\subsection{Violation One: Nondiscrimination Policy}

The first and foremost ethical implication of the Great Boston Molasses Flood was the location of construction. Located in the North End of Boston, then primarily inhabited by Sicilians and other Southern Italians, the United States Industrial Alcohol company planned construction due to the lack of political influence of the region \cite{b1}. This violates the IEEE Code of Ethics, section II, part 7, which states ``to treat all persons fairly and with respect, and to not engage in discrimination based on characteristics such as race, religion, gender, disability, age, national origin, sexual orientation, gender identity, or gender expression.'' As such, the fact that the location was selected primarily due to the lack of political power of people of a certain national origin is indicative of a code violation.

\subsection{Violation Two: Public Welfare}

Another ethical implication is that of public safety; whereas most, if not all, professional engineering organizations, including IEEE, ``hold paramount the safety, health, and welfare of the public\dots'' (Code of Ethics, section I, part 1) \cite{b3}, then Purity Distilling Company Treasurer Arthur P. Jell did not. His choices were solely based on both personal and company benefit, as it was made clear to him that he was not to be promoted if he did not finish the project by the new year. Additionally, if he were not to complete the project by said deadline, the Unites States Industrial Alcohol company would lose a significant portion of money, as a ship inbound from Cuba would not have anywhere to unload. Thus, by expediting the project without regard for public safety, especially given its proximity to a residential area, Jell and USIA did not follow IEEE ethical guidelines. 

\subsection{Violation Three: Failure to Fix Mistakes}

A third ethical implication is Jell and his subordinate's failure to recognize and fix the mistakes. Isaac Gonzalez, a worker at the molasses tank site, noticed major leaks and pieces of broken, rusted steel, which he brought to his boss, and his boss's boss, Jell \cite{b1}. Both people disregarded this, calling Gonzalez paranoid and overly-cautious. The maximum extent of their response was painting the tank brown, which would prevent the colors of the rust, as well as the leaking brown molasses, from being visible on the outside. Additionally, Gonzalez was threatened with losing his job if he was to continue bringing the issues up. In this manner, section I, part 5, which states ``to seek, accept, and offer honest criticism of technical work, to acknowledge and correct errors\dots'' and section III, part 10, which states, ``to support colleagues and co-workers in following this code of ethics, to strive to ensure the code is upheld, and to not retaliate against individuals reporting a violation.'' \cite{b3} were violated.

\subsection{Violation Four: Property Damage through Negligence}

Finally, USIA violated section II, part 9, which states ``to avoid injuring others, their property, \dots or employment by false or malicious actions, rumors or any other verbal or physical abuses'' \cite{b3}. Firstly, they violated this by threatening to fire Isaac Gonzalez (who was not even unionized), which would leave him without a job or welfare. Secondly, the negligence as related to Gonzalez's grievances led to significant property damage, valued at over \$100 million. This is in direct violation of this part of the IEEE Code of Ethics.

\subsection{Summary}

As is evident, the conflict of interest, as well as just human greed, led to the disaster now known as the Great Boston Molasses Flood. By adhering to the aforementioned, modernized guidelines, many deaths and lots of property damage could have been prevented. To this day, if the weather is warm enough, the streets of Boston might smell just a bit like molasses.

\section{Chernobyl}

\subsection{Background}

The disaster at the Chernobyl nuclear power plant occurred in 1986, when there was a test of an emergency water cooling system. The previous day, the reactor was set to about 7\% of its power generation capabilities, and controlling graphite rods were withdrawn. When the test was performed at 1:23 am, reactions began to occur at an uncontrollable rate, resulting in a quick build up of steam. The steam kept building until, due to the extreme pressure, the 1000 ton roof of Reactor 4 exploded, and a partial reactor meltdown occurred. Seconds later, a second explosion occurred, leading to fires on the roof of Reactor 3. Fires and wind currents would lead to significant dissemination of radioactive materials throughout the surrounding areas, reaching as far as Sweden \cite{b2}.

\subsection{Violation One: Cover-ups}

Soviet officials attempted to cover up the disaster, as they delayed any kind of international announcement. It wouldn't be until Swedish scientists measured increased radiation in the air that they demanded information from the Soviets. Still, Soviet officials tried to downplay the significance of the event, making it seem like a small accident. Even though the disaster occurred very early in the morning of April \nth{26}, it wouldn't be until April \nth{28} that an international statement was issued \cite{b2}. As such, this is in direct violation of section I, part 1 of the IEEE Code of Ethics, which relates to public welfare being the first and foremost priority of any engineering endeavor \cite{b3}. Additionally, section I, part 4 is violated. This part states, ``to avoid unlawful conduct in professional activities'' \cite{b3}. Nondisclosure of an accident of such scale is clearly unlawful conduct, due to the magnitude of people endangered. Furthermore, the section II, part 9, which concerns avoiding injury to people or their property is clearly violated, as, by not disclosing what had happened, the people in Prip'yat were put at a significant risk, as indicated by the amount of cancer-related deaths which occurred within a short amount of time of the disaster. Additionally, the cost of clean up is estimated to have cost around \$68 billion, when adjusted to 2019 inflation.

\subsection{Adherence One: Performed by Professionals}

Section I, part 6 of the IEEE Code of Ethics \cite{b3} states that it as an electrical engineer's duty ``to maintain and improve our technical competence and to undertake technological tasks for others only if qualified by training or experience, or after full disclosure of pertinent limitations''. The actual testing of the RBMK unit was performed by scientists, whose education made them professionals capable of performing such an evaluation of a nuclear reactor; it is ultimately the poor construction and limitations of available tools that led to the disaster. As such, the disaster did not entirely involve ethical violations, but rather some ethical codes were observed. 

\subsection{Modern Relevance}

The IEEE Code of Ethics was initially adopted in 1974, and then revised in 1979 (and later in 1987). As such, it is a possibility that some of the codes that were observed were a result of IEEE publications, and thus, goes to show the benefits of such developments.

\begin{thebibliography}{00}
\bibitem{b1} E. P. Brunet, Jr., “Engineering Ethics: The Great Boston Molasses Flood.”
\bibitem{b2} H. com Editors, “Chernobyl,” HISTORY. \href{https://www.history.com/topics/1980s/chernobyl#what-happened-at-chernobyl}{https://www.history.com/topics/1980s/chernobyl#what-happened-at-chernobyl}
\bibitem{b3} “IEEE Code of Ethics,” IEEE. [Online]. Available: \href{https://www.ieee.org/about/corporate/governance/p7-8.html}{https://www.ieee.org/about/corporate/governance/p7-8.html}. [Accessed: 06-Dec-2022]. 
\end{thebibliography}

\end{document}
