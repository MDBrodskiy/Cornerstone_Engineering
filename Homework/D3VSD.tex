\documentclass[conference]{IEEEtran}
\IEEEoverridecommandlockouts
% The preceding line is only needed to identify funding in the first footnote. If that is unneeded, please comment it out.
\usepackage{cite}
\usepackage{amsmath,amssymb,amsfonts}
\usepackage{algorithmic}
\usepackage{graphicx}
\usepackage{textcomp}
\usepackage{xcolor}
\usepackage{hyperref}
%\hypersetup{
%colorlinks=true,
%linkcolor=blue,
%filecolor=magenta,      
%urlcolor=blue,
%citecolor=blue,
%}
\def\BibTeX{{\rm B\kern-.05em{\sc i\kern-.025em b}\kern-.08em
    T\kern-.1667em\lower.7ex\hbox{E}\kern-.125emX}}
\begin{document}

\title{D3 — Value Sensitive Design}

\author{\IEEEauthorblockN{Michael Brodskiy}
\IEEEauthorblockA{\textit{College of Engineering} \\
\textit{Northeastern University}\\
Boston, U.S.A \\
\href{mailto:Brodskiy.M@Northeastern.edu}{Brodskiy.M@Northeastern.edu}}
%\and
%\IEEEauthorblockN{2\textsuperscript{nd} Given Name Surname}
%\IEEEauthorblockA{\textit{dept. name of organization (of Aff.)} \\
%\textit{name of organization (of Aff.)}\\
%City, Country \\
%email address or ORCID}
%\and
%\IEEEauthorblockN{3\textsuperscript{rd} Given Name Surname}
%\IEEEauthorblockA{\textit{dept. name of organization (of Aff.)} \\
%\textit{name of organization (of Aff.)}\\
%City, Country \\
%email address or ORCID}
%\and
%\IEEEauthorblockN{4\textsuperscript{th} Given Name Surname}
%\IEEEauthorblockA{\textit{dept. name of organization (of Aff.)} \\
%\textit{name of organization (of Aff.)}\\
%City, Country \\
%email address or ORCID}
%\and
%\IEEEauthorblockN{5\textsuperscript{th} Given Name Surname}
%\IEEEauthorblockA{\textit{dept. name of organization (of Aff.)} \\
%\textit{name of organization (of Aff.)}\\
%City, Country \\
%email address or ORCID}
%\and
%\IEEEauthorblockN{6\textsuperscript{th} Given Name Surname}
%\IEEEauthorblockA{\textit{dept. name of organization (of Aff.)} \\
%\textit{name of organization (of Aff.)}\\
%City, Country \\
%email address or ORCID}
}

\maketitle

\begin{abstract}
  This document outlines an instance on the Northeastern University Boston campus where value-sensitive design was not taken into consideration; furthermore, it provides a possible solution to said design mistake, in addition to restating the design as reformed problem statement.
\end{abstract}

\begin{IEEEkeywords}
  technical, success, design, value-sensitive, technology, access
\end{IEEEkeywords}

\section{Introduction}
This document is prepared in accordance with the Institute of Electrical and Electronics Engineers (IEEE) specifications, and attempts to identify a value-sensitive design problem.

\section{The Problem}

\subsection{Description}

\subsection{Significance}

\subsection{Defining Success}

\section{Value-Sensitive Design}

\subsection{Incorporating Value-Sensitive Design}

\subsection{Byproducts of the Aforementioned Solution}

\subsection{Employing Technology to Mitigate Byproducts}

\section{Redefined Problem Statement}

\end{document}
